\label{sec:ifc-meta}

An IFC file may include partitions containing descriptions about the tools producing that file or other metadescription of the file.


\section{Producing Tool}
\label{sec:ifc-tool-maker}

A tool producing an IFC may embed a partition identifying that tool along with other information about that tool. 
That partition is called the \emph{tool maker} partition.
Each entry in that partition is a \type{TextOffset} representing a NUL-terminated UTF-8 encoded string (see \secref{sec:ifc-textoffset-data-type}).  
The partition has as many entries as the tool deems necessary for identification purposes.

\partition{meta.tool.maker}

\section{Tool Invocation}
\label{sec:ifc-tool-invocation}

A tool producing an IFC file may embed a partition describing the sequence of invocations of tools leading to the creating of that file.
That partition is called the \emph{tool invocation} partition.  Each entry in that partition is
a structure of the layout
%
\begin{figure}[H]
    \centering
    \structure{
        \DeclareMember{cmd}{TextOffset} \\
        \DeclareMember{args}{Sequence TextOffset} \\
    }
    \caption{Structure of a tool invocation}
    \label{fig:ifc-tool-invocation}
\end{figure}
%
with the following meaning:
\begin{itemize}
    \item The \field{cmd} field describes the name of the invoked command
    \item The \field{args} field describes the sequence (\secref{sec:ifc-sequence}) of arguments to the 
    invoked command.  It is a slice of the string heap (\secref{sec:ifc-string-heap}), meaning that the field \field{args.start}
    designates an entry into the \emph{string heap} and \field{args.cardinality} designates how many 
    \type{TextOffset} values there are in that sequence.  Each element of that sequence designates a NUL-terminated UTF-8 string.
\end{itemize}


\partition{meta.tool.invocation}

