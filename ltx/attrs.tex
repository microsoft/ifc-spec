\label{sec:ifc-attrs}


Attributes are denoted by attribute references of type \type{AttrIndex}. 
Like all abstract references, an \type{AttrIndex} value is 32-bit wide with alignment $4$.
\begin{figure}[htbp]
  \centering
  \absref{4}{AttrSort}
  \caption{\type{AttrIndex}: Abstract reference of attribute}
  \label{fig:ifc-attr-index}
\end{figure}

\begin{SortEnum}{AttrSort}
    \enumerator{Nothing}
    \enumerator{Basic}
    \enumerator{Scoped}
    \enumerator{Labeled}
    \enumerator{Called}
    \enumerator{Expanded}
    \enumerator{Factored}
    \enumerator{Elaborated}
    \enumerator{Tuple}
\end{SortEnum}

When an input source file contains a C++ attribute (with \type{AttrIndex} value \var{a})
on a declaration (with \type{DeclIndex} value \var{d}), an associative pair $(\var{d}, \var{a})$
is entered into the \code{".msvc.trait.decl-attrs"} partition 
(\secref{sec:ifc-msvc-decl-attr-trait}).

\section{Attribute structures}
\label{sec:ifc:AttrSort-structures}

\subsection{\valueTag{AttrSort::Nothing}}
\label{sec:ifc:AttrSort:Nothing}

An \type{AttrIndex} value with tag \valueTag{AttrSort::Nothing} indicates an empty attribute
at the source level, e.g. ``\code{[[]]}''.

No partition is emitted for attributes of this sort.

\subsection{\valueTag{AttrSort::Basic}}
\label{sec:ifc:AttrSort:Basic}

An \type{AttrIndex} value with tag \valueTag{AttrSort::Basic} denotes a word (\secref{sec:ifc-words}) used in an attribute,
for example \code{answer} and \code{42} in ``\code{[[answer(42)]]}'' are basic attributes.
The \field{index} field of that abstract reference is an index into the basic attributes partitions.
Each entry in that partition is a structure with the following layout
%
\begin{figure}[H]
    \centering
    \structure{
        \DeclareMember{word}{Word} \\
    }
    \caption{Structure of a basic attribute}
    \label{fig:ifc:AttrSort:Basic}
\end{figure}
%
The \field{word} field represents that word making up this basic attribute.

\partition{attr.basic}


\subsection{\valueTag{AttrSort::Scoped}}
\label{sec:ifc:AttrSort:Scoped}

An \type{AttrIndex} reference with tag \valueTag{AtrSort::Scoped} denotes a scoped attribute, that is a
basic attribute followed by ``\code{::}'' and another basic attribute.
For example, \code{gsl::suppress} in ``\code{[[gsl::suppress(type.1)]]}'' is a scoped attribute.
The \field{index} field of that abstract reference is an index into the scoped attribute partition.
Each entry in that partition is a structure with the following layout
%
\begin{figure}[H]
    \centering
    \structure{
        \DeclareMember{scope}{Word} \\
        \DeclareMember{member}{Word} \\
    }
    \caption{Structure of a scoped attribute}
    \label{fig:ifc:AttrSort:Scoped}
\end{figure}
%
The \field{scope} field represents the word appearing before the ``\code{::}'' token.
The \field{member} field represents the word appearing after the ``\code{::}'' token.

\partition{attr.scoped}

\subsection{\valueTag{AttrSort::Labeled}}
\label{sec:ifc:AttrSort:Labeled}

An \type{AttrIndex} reference with tag \valueTag{AtrSort::labeled} denotes a labeled attribute of the form ``\grammar{key} \code{:} \grammar{value}''.  For example,
while an attribute-like syntax ``\code{[[requires: x > 0]]}'' is not yet a standard C++ conformant, it has been proposed for use by contract conditions.  Furthermore, C++11 attributes in called 
attributes can use labeled attributes to provide key-value pairs, as in ``\code{[[revert(commit: 0xdeadbeef, reason: "bonkers")]]}''.

The \field{index} field of that abstract reference is an index into the labeled attribute partition.  Each entry in that partition is a structure
with the following layout
%
\begin{figure}[H]
    \centering
    \structure{
        \DeclareMember{label}{Word} \\
        \DeclareMember{attribute}{AttrIndex} \\
    }
    \caption{Structure of a labeled attribute}
    \label{fig:ifc:AttrSort:Labeled}
\end{figure}
%
The \field{label} field represents the word used as \grammar{key} in this attribute.
The \field{attribute} field denotes the \grammar{value} in this attribute.

\partition{attr.labeled}

\subsection{\valueTag{AttrSort::Called}}
\label{sec:ifc:AttrSort:Called}

An \type{AttrIndex} reference with tag \valueTag{AttrSort::Called} denotes an attribute using a syntax similar to that of function call.  For example,
``\code{[[deprecated("out of fad")]]}'' and ``\code{[[gsl::suppress(type.1)]]}'' are called attributes.

The \field{index} field of that abstract reference is an index into the called attributes partition.  Each entry in
that partition is a structure with the following layout
%
\begin{figure}[H]
    \centering
    \structure{
        \DeclareMember{function}{AttrIndex} \\
        \DeclareMember{arguments}{AttrIndex} \\
    }
    \caption{Structure of a called attribute}
    \label{fig:ifc:AttrSort:Called}
\end{figure}
%
The \field{function} field denotes the attribute appearing in function position, whereas the \field{arguments} denotes
the sequence of attributes making up the supplied argument list.

\partition{attr.called}

\subsection{\valueTag{AttrSort::Expanded}}
\label{sec:ifc:AttrSort:Expanded}

An \type{AttrIndex} reference with tag \valueTag{AttrSort::Expanded} denotes an attribute using the ellipsis after an attribute, in a syntax similar to that of pack expansion.
For example, ``\code{[[fun(args)...]]}'' is an expanded attribute.

The \field{index} field of that abstract reference is an index into the expanded attributes partition.
Each entry in that partition is a structure with the following layout
%
\begin{figure}[H]
    \centering
    \structure{
        \DeclareMember{operand}{AttrIndex} \\
    }
    \caption{Structure of an expanded attribute}
\end{figure}
%
The \field{operand} field denotes the attribute to be expanded.

\partition{attr.expanded}

\subsection{\valueTag{AttrSort::Factored}}
\label{sec:ifc:AttrSort:Factored}

An \type{AttrIndex} reference with tag \valueTag{AttrSort::Factored} denotes an abbreviation form of attribute
sequence where each attribute is either a scoped attribute (\sortref{Scoped}{AttrSort}) 
or a called attribute (\sortref{Called}{AttrSort}) where the attribute in function position 
is a scoped attribute, and all mentioning the same scope word.  For example, 
``\code{[[using build: opt(2), chk]]}'' is a factored attribute, and is a short-hand for 
``\code{[[build::opt(2), build::chk]]}''.

The \field{index} field of that abstract reference is an index into the factored attributes partition.
Each entry in that partition is a structure with the following layout
%
\begin{figure}[H]
    \centering
    \structure{
        \DeclareMember{factor}{Word} \\
        \DeclareMember{terms}{AttrIndex} \\
    }
\end{figure}
%
The \field{factor} field represents the common scope word.  The field \field{terms} represents the 
rest of the comm-separated list of attributes.

\partition{attr.factored}

\subsection{\valueTag{AttrSort::Elaborated}}
\label{sec:ifc:AttrSort:Elaborated}

An \type{AttrIndex} reference with tag \valueTag{AttrSort::Elaborated} denotes an attribute 
where the input source-level tokens have been parsed and semantically analyzed, and the resulting
expression embedded in the attribute structure.  This functionality supports various popular
implementation-defined extensions, and also proposed contract-attribute syntaxes such as
``\code{[[assert: condition]]}'' where \grammar{condition} is a Boolean expression stating an
expectation at a given program point.

The \field{index} field of that abstract reference is an index into the elaborated attribute partition.
Each entry in that partition is a structure with the following layout
%
\begin{figure}[H]
    \centering
    \structure{
        \DeclareMember{expression}{ExprIndex} \\
    }
    \caption{Structure of an elaborated attribute}
    \label{fig:ifc:AttrSort:Elaborated}
\end{figure}
%
The \field{expression} field denotes the result of the parsing and semantic analysis of the 
source-level words making up the attributes.  The input source level words are not retained.

\partition{attr.elaborated}

\subsection{\valueTag{AttrSort::Tuple}}
\label{sec:ifc:AttrSort:Tuple}

An \type{AttrIndex} reference with tag \valueTag{AttrSort::Tuple} denotes a sequence of comma-separated attributes.
It is a general scheme to make a sequence of attributes appear as if it was just an attribute.  This representation
brings simplicity, flexibility, and generality to the attribute data structures.

The \field{index} field of that abstract reference is an index into the tuple attribute partition.
Each entry in that partition is a structure with the following layout
%
\begin{figure}[H]
    \centering
    \structure{
        \DeclareMember{start}{Index} \\
        \DeclareMember{cardinality}{Cardinality} \\
    }
    \caption{Structure of a tuple attributes}
    \label{fig:ifc:AttrSort:Tuple}
\end{figure}
The \field{start} designates the position of the first \type{AttrIndex} value in the 
attribute heap (\secref{sec:ifc-attr-heap}) corresponding to the first abstract reference in the 
tuple attribute.
The \field{cardinality} represents the number of \type{AttrIndex} values in the tuple attribute.
If this value is zero, then \field{start} is not meaningful.

\partition{attr.tuple}
