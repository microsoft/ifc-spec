\label{sec:ifc-names}

Names are indicated by abstract references.  This document
uses \type{NameIndex} to designate a typed abstract reference to a
name.  Like all abstract references, it is a 32-bit value
\begin{figure}[H]
  \centering
	\absref{3}{NameSort}
  \caption{\type{NameIndex}: Abstract reference of names}
  \label{fig:ifc-name-index}
\end{figure}

\begin{SortEnum}{NameSort}
	\enumerator{Identifier}
	\enumerator{Operator}
	\enumerator{Conversion}
	\enumerator{Literal}
	\enumerator{Template}
	\enumerator{Specialization}
	\enumerator{SourceFile}
	\enumerator{Guide}
\end{SortEnum}

\noindent
In most cases, the \field{index} field of a \type{NameIndex} is a numerical index into a partition (\secref{sec:ifc-partition}) denoted by the \field{tag} field.

\section{Name structures}
\label{sec:ifc:name-structures}

\subsection{\valueTag{NameSort::Identifier}}
\label{sec:ifc:NameSort:Identifier}

A \type{NameIndex} value with tag \valueTag{NameSort::Identifier} represents an abstract reference for normal alphabetic identifiers.
The \field{index} field in this case is also the structure representation of the identifier: it is conceptually a value of 
type \type{TextOffset}, represented with $29$ bits.  It is an index into the string table (\secref{sec:ifc-string-table}).
\begin{figure}[H]
  \centering
	\structure{
		{\field{tag} = \clipValue{NameSort::Identifier}{3}} \\
		{\field{value} : \clipValue{\type{TextOffset}}{29}} \\
	}
  \caption{Structure of a \type{NameIndex} representing an identifier}
  \label{fig:ifc-identifier-structure}
\end{figure}

\note{Identifiers do not have a separate dedicated partition.  Rather, identifiers are stored in the IFC's string table.}
%\partition{name.identifier}

\subsection{\valueTag{NameSort::Operator}}
\label{sec:ifc:NameSort:Operator}

A \type{NameIndex} value with tag \valueTag{NameSort::Operator} represents an operator function name.  
The \field{index} field is an index into the operator partition (\secref{sec:ifc-partition}).  
Each entry in that partition has two components: a \field{category} field denoting the specified operator, and an \field{encoded} field.
\begin{figure}[H]
	\centering
	\structure{
		\DeclareMember{encoded}{TextOffset} \\
		\DeclareMember{operator}{Operator} \\
	}
	\caption{Structure of an operator-function name}
	\label{fig:ifc-operator-function-structure}
\end{figure}
%
The \field{encoded} field is a text encoding of the operator function name.
The \field{operator} field is a 16-bit value (\secref{sec:ifc-operators}) 
capable of denoting any operator used in a valid C++ program. 

\partition{name.operator}

\subsection{\valueTag{NameSort::Conversion}}
\label{sec:ifc:NameSort:Conversion}

A \type{NameIndex} value with tag \valueTag{NameSort::Conversion} represents a conversion-function name.
The \field{index} field is an index into the conversion function name partition.
Each entry in that partition is a structure with two components.
\begin{figure}[H]
	\centering
	\structure{
		\DeclareMember{target}{TypeIndex} \\
		\DeclareMember{encoded}{TextOffset} \\
	}
	\caption{Structure of a conversion-function name}
	\label{fig:ifc-conversion-function-id-structure}
\end{figure}
%
The \field{target} field is an abstract reference to the target type (\secref{sec:ifc-types}) of the conversion function.
The \field{encoded} field designates a text encoding of the operator function name.

\note{This data structure is subject to change}

\partition{name.conversion}

\subsection{\valueTag{NameSort::Literal}}
\label{sec:ifc:NameSort:Literal}

A \type{NameIndex} value with tag \valueTag{NameSort::Literal} represents a reference to a string literal operator name.
The \field{index} field is an index into the string literal operator partition (\secref{sec:ifc-partition}). Each element in that
partition has two components: a \field{suffix} field, and an \field{encoded} field. 
%
\begin{figure}[H]
	\centering
	\structure{
%		\DeclareMember{suffix}{TextOffset} \\
		\DeclareMember{encoded}{TextOffset} \\
	}
	\caption{Structure of a literal-operator name}
	\label{fig:ifc-literal-operator-name-structure}
\end{figure}
% The \field{suffix} field is an index into the string table (\secref{sec:string-table}) and denotes the suffix in the declaration of the string literal operator.
The \field{encoded} field is an index into the string table, and represents the text encoding of the literal operator.

\note{Previous versions of this document showed a \field{suffix} for the literal operator name.  That did not match recent behavior of the VC++ compiler.  That field will be added back in the future.}

\partition{name.literal}

\subsection{\valueTag{NameSort::Template}}
\label{sec:ifc:NameSort:Template}

A \type{NameIndex} value with tag \valueTag{NameSort::Template} represents a reference
to an assumed (as opposed to \emph{declared}) template name.  This is the case of nested-name of qualified-id where
the qualifier is a dependent name and the unqualified part is asserted to name a template.
The \field{index} field is an index into the partition of assumed template names.
%
\begin{figure}[H]
	\centering
	\structure{
		\DeclareMember{name}{NameIndex} \\
	}
	\caption{Structure of an assumed template name}
	\label{fig:ifc-assumed-template-name-structure}
\end{figure}
%
Each entry in that partition is a structure with exactly one field, \field{name}, that is itself a \type{NameIndex}. 
It is an error for the \field{tag} field of \field{name} to be \valueTag{NameSort::Template}.

\partition{name.template}

\subsection{\valueTag{NameSort::Specialization}}
\label{sec:ifc:NameSort:Specialization}

A \type{NameIndex} value with tag \valueTag{NameSort::Specialization} represents a reference
to a template-id, i.e. what in C++ source code is a template-name followed by a template-argument list.
The \field{index} field is an index into the template-id partition.  Each entry in that partition has two components:
a \field{primary} field, and an \field{arguments} field.
%
\begin{figure}[H]
	\centering
	\structure{
		\DeclareMember{primary}{NameIndex} \\
		\DeclareMember{arguments}{ExprIndex} \\
	}
	\caption{Structure of a template-id name}
	\label{fig:ifc-template-id-structure}
\end{figure}
%
The \field{primary} field represents the name of the primary template.
The \field{arguments} field represents the template argument list as an expression (\secref{sec:ifc-exprs}).
If the template-argument list is empty or a singleton, the \field{arguments} field an abstraction reference
to that expression.  Otherwise, the \field{arguments} field is a tuple expression.

\partition{name.specialization}

\subsection{\valueTag{NameSort::SourceFile}}
\label{sec:ifc:NameSort:SourceFile}

A \type{NameIndex} value with tag \valueTag{NameSort::SourceFile} represents a reference
to a source file name. The \field{index} field is an index into the partition of source file names.
Each entry in that partition has two components: a \field{path} field, and a \field{guard} field.
%
\begin{figure}[H]
	\centering
	\structure{
		\DeclareMember{path}{TextOffset} \\
		\DeclareMember{guard}{TextOffset} \\
	}
	\caption{Structure of a source file name}
	\label{fig:ifc-source-file-name-structure}
\end{figure}
%
The \field{path} field is an index into the string table.
The \field{guard} field is also an index into the string table, and represents the identifier of the source-level
include guard of the file, if it has any.

\partition{name.source-file}

\subsection{\valueTag{NameSort::Guide}}
\label{sec:ifc:NameSort:Guide}

A \type{NameIndex} value with tag \valueTag{NameSort::SourceFile} represents a reference to a 
user-authored deduction guide name for a class template.
Note that deduction guides don't have names at the C++ source level.  
The \field{index} field is an index into the deduction guides partition.  
Each entry in that partition has one component: a \type{DeclIndex} designating the 
primary (class) template ({sec:ifc:DeclSort:Template}):
%
\begin{figure}[H]
	\centering
	\structure{
		\DeclareMember{primary\_template}{DeclIndex} \\
	}
	\caption{Structure of a deduction guide name}
	\label{fig:ifc-guide-name-structure}
\end{figure} 

\partition{name.guide}


