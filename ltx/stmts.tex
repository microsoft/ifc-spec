\label{sec:ifc-statements}

\paragraph{Note: All data structures described in this chapter are subject to change.}

Statements are represented in an IFC as part of the body of constexpr or inline function defined in a
module interface source file.  A statement is designated an abstract reference of type \type{StmtIndex}:
\begin{figure}[H]
  \centering
  \absref{5}{StmtSort}
  \caption{\type{StmtIndex}: Abstract reference of statement}
  \label{fig:ifc-stmt-index}
\end{figure}

\begin{SortEnum}{StmtSort}
	\enumerator{VendorExtension}
	\enumerator{Empty}
	\enumerator{If}
	\enumerator{For}
	\enumerator{Case}
	\enumerator{While}
	\enumerator{Block}
	\enumerator{Break}
	\enumerator{Switch}
	\enumerator{DoWhile}
	\enumerator{Default}
	\enumerator{Continue}
	\enumerator{Expression}
	\enumerator{Return}
	\enumerator{VariableDecl}
	\enumerator{Expansion}
	\enumerator{SyntaxTree}
\end{SortEnum}

\section{Statement structures}
\label{sec:ifc-ststement-structures}

\subsection{\valueTag{VendorExtension}}
\label{sec:ifc:StmtSort:VendorExtension}

\partition{stmt.vendor-extension}


\subsection{\valueTag{StmtSort::Empty}}
\label{sec:ifc:StmtSort:Empty}

\begin{figure}[H]
	\centering
	\structure{
		\DeclareMember{locus}{SourceLocation} \\
	}
	\caption{Structure of an empty statement}
	\label{fig:ifc-empty-stmt-structure}
\end{figure}


\partition{stmt.empty}


\subsection{\valueTag{StmtSort::If}}
\label{sec:ifc:StmtSort:If}

\begin{figure}[H]
	\centering
	\structure{
		\DeclareMember{initialization}{StmtIndex} \\
		\DeclareMember{condition}{StmtIndex} \\
		\DeclareMember{consequence}{StmtIndex} \\
		\DeclareMember{alternative}{StmtIndex} \\
		\DeclareMember{locus}{SourceLocation} \\
	}
	\caption{Structure of an if-statement}
	\label{fig:ifc-if-stmt-structure}
\end{figure}

\partition{stmt.if}


\subsection{\valueTag{StmtSort::For}}
\label{sec:ifc:StmtSort:For}

\begin{figure}[H]
	\centering
	\structure{
		\DeclareMember{initialization}{StmtIndex} \\
		\DeclareMember{condition}{StmtIndex} \\
		\DeclareMember{continuation}{StmtIndex} \\
		\DeclareMember{body}{StmtIndex} \\
		\DeclareMember{locus}{SourceLocation} \\
	}
	\caption{Structure of a for-statement}
	\label{fig:ifc-for-stmt-structure}
\end{figure}

\partition{stmt.for}


\subsection{\valueTag{StmtSort::Case}}
\label{sec:ifc:StmtSort:Case}

\begin{figure}[H]
	\centering
	\structure{
		\DeclareMember{expr}{ExprIndex} \\
		\DeclareMember{locus}{SourceLocation} \\
	}
	\caption{Structure of a case-label}
	\label{fig:ifc-case-label-structure}
\end{figure}

\partition{stmt.case}



\subsection{\valueTag{StmtSort::While}}
\label{sec:ifc:StmtSort:While}

\begin{figure}[H]
	\centering
	\structure{
		\DeclareMember{condition}{StmtIndex} \\
		\DeclareMember{body}{StmtIndex} \\
		\DeclareMember{locus}{SourceLocation} \\
	}
	\caption{Structure of a while-statement}
	\label{fig:ifc-while-stmt-structure}
\end{figure}

\partition{stmt.while}


\subsection{\valueTag{StmtSort::Block}}
\label{sec:ifc:StmtSort:Block}

\begin{figure}[H]
	\centering
	\structure{
		\DeclareMember{start}{Index} \\
		\DeclareMember{cardinality}{Cardinality} \\
	}
	\caption{Structure of a block statement}
	\label{fig:ifc-block-stmt-structure}
\end{figure}

\partition{stmt.block}


\subsection{\valueTag{StmtSort::Break}}
\label{sec:ifc:StmtSort:Break}

\begin{figure}[H]
	\centering
	\structure{
		\DeclareMember{locus}{SourceLocation} \\
	}
	\caption{Structure of a break statement}
	\label{fig:ifc-break-stmt-structure}
\end{figure}

\partition{stmt.break}



\subsection{\valueTag{StmtSort::Switch}}
\label{sec:ifc:StmtSort:Switch}

\begin{figure}[H]
	\centering
	\structure{
		\DeclareMember{initialization}{StmtIndex} \\
		\DeclareMember{condition}{ExprIndex} \\
		\DeclareMember{body}{StmtIndex} \\
		\DeclareMember{locus}{SourceLocation} \\
	}
	\caption{Structure of a switch-statement}
	\label{fig:ifc-switch-stmt-structure}
\end{figure}

\partition{stmt.switch}

\subsection{\valueTag{StmtSort::DoWhile}}
\label{sec:ifc:StmtSort:DoWhile}

\begin{figure}[H]
	\centering
	\structure{
		\DeclareMember{condition}{StmtIndex} \\
		\DeclareMember{body}{StmtIndex} \\
		\DeclareMember{locus}{SourceLocation} \\
	}
	\caption{Structure of a do-statement}
	\label{fig:ifc-do-stmt-structure}
\end{figure}

\partition{stmt.do-while}


\subsection{\valueTag{StmtSort::Default}}
\label{sec:ifc:StmtSort:Default}

\begin{figure}[H]
	\centering
	\structure{
		\DeclareMember{locus}{SourceLocation} \\
	}
	\caption{Structure of a default label}
	\label{fig:ifc-default-label-structure}
\end{figure}

\partition{stmt.default}


\subsection{\valueTag{StmtSort::Continue}}
\label{sec:ifc:StmtSort:Continue}

\begin{figure}[H]
	\centering
	\structure{
		\DeclareMember{locus}{SourceLocation} \\
	}
	\caption{Structure of a continue statement}
	\label{fig:ifc-continue-stmt-structure}
\end{figure}

\partition{stmt.continue}

\subsection{\valueTag{StmtSort::Expression}}
\label{sec:ifc:StmtSort:Expression}

\begin{figure}[H]
	\centering
	\structure{
		\DeclareMember{expr}{ExprIndex} \\
		\DeclareMember{locus}{SourceLocation} \\
	}
	\caption{Structure of an expression-statement}
	\label{fig:ifc-expr-stmt-structure}
\end{figure}

\partition{stmt.expression}

\subsection{\valueTag{StmtSort::Return}}
\label{sec:ifc:StmtSort:Return}

\begin{figure}[H]
	\centering
	\structure{
		\DeclareMember{expr}{ExprIndex} \\
		\DeclareMember{function\_type}{TypeIndex} \\
		\DeclareMember{expression\_type}{TypeIndex} \\
		\DeclareMember{locus}{SourceLocation} \\
	}
	\caption{Structure of a return statement}
	\label{fig:ifc-return-stmt-structure}
\end{figure}

\partition{stmt.return}

\subsection{\valueTag{StmtSort::VariableDecl}}
\label{sec:ifc:StmtSort:VariableDecl}

\begin{figure}[H]
	\centering
	\structure{
		\DeclareMember{decl}{DeclIndex} \\
		\DeclareMember{initializer}{ExprIndex} \\
		\DeclareMember{locus}{SourceLocation} \\
	}
	\caption{Structure of a variable-declaration statement}
	\label{fig:ifc-decl-stmt-structure}
\end{figure}

\partition{stmt.variable}

\subsection{\valueTag{StmtSort::Expansion}}
\label{sec:ifc:StmtSort:Expansion}

A \type{StmtIndex} abstract reference with tag \valueTag{StmtSort::Expansion}
designates a statement expansion.

\begin{figure}[H]
	\centering
	\structure{
		\DeclareMember{operand}{StmtIndex} \\
	}
	\caption{Structure of an expansion statement}
	\label{fig:ifc-expansion-stmt-structure}
\end{figure}

\partition{stmt.expansion}

\subsection{\valueTag{StmtSort::SyntaxTree}}
\label{sec:ifc:StmtSort:SyntaxTree}

\begin{figure}[H]
	\centering
	TBD
\end{figure}

\partition{stmt.syntax-tree}


