\label{sec:ifc-statements}

\paragraph{Note: All data structures described in this chapter are subject to change.}

Statements are represented in an IFC as part of the body of constexpr or inline function defined in a
module interface source file.  A statement is designated an abstract reference of type \type{StmtIndex}:
\begin{figure}[H]
  \centering
  \absref{5}{StmtSort}
  \caption{\type{StmtIndex}: Abstract reference of statement}
  \label{fig:ifc-stmt-index}
\end{figure}

\begin{SortEnum}{StmtSort}
	\enumerator{VendorExtension}
	\enumerator{Try}
	\enumerator{If}
	\enumerator{For}
	\enumerator{Labeled}
	\enumerator{While}
	\enumerator{Block}
	\enumerator{Break}
	\enumerator{Switch}
	\enumerator{DoWhile}
	\enumerator{Goto}
	\enumerator{Continue}
	\enumerator{Expr}
	\enumerator{Return}
	\enumerator{Decl}
	\enumerator{Expansion}
	\enumerator{SyntaxTree}
	\enumerator{Handler}
	\enumerator{Tuple}
	\enumerator{Directive}
\end{SortEnum}

\section{Statement structures}
\label{sec:ifc-ststement-structures}

\subsection{\valueTag{VendorExtension}}
\label{sec:ifc:StmtSort:VendorExtension}

\partition{stmt.vendor-extension}


\subsection{\valueTag{StmtSort::Try}}
\label{sec:ifc:StmtSort:Try}

A \type{StmtIndex} value with tag \valueTag{StmtSort::Try} designates a \grammar{try-block} statement.
The \field{index} field of that abstract reference is an index into the try-block partition.
Each entry in that partition is a structure with the following layout
%
\begin{figure}[H]
	\centering
	\structure{
		\DeclareMember{locus}{SourceLocation} \\
		\DeclareMember{start}{Index} \\
		\DeclareMember{cardinality}{Cardinality} \\
		\DeclareMember{handlers}{StmtIndex} \\
	}
	\caption{Structure of a \grammar{try-block}}
	\label{fig:ifc-try-stmt-structure}
\end{figure}
%
The fields have the following meanings:
\begin{itemize}
	\item the sequence of statements making up the \grammar{compound-statement} body of the \grammar{try-block} is described by a \type{Sequence}
	      of \type{StmtIndex} abstract references stored in the statement index heap (\secref{sec:ifc-stmt-heap}).
		  \begin{itemize}
		     \item \field{start} is the index of the entry in the statement heap for the first \type{StmtIndex} abstract reference of the body of the \grammar{try-block}
		     \item \field{cardinality} denotes the number of \type{StmtIndex} abstract references in the body of the \grammar{try-block}.
		  \end{itemize}
	\item \field{handlers} designates either the handler or the sequence of handlers (\sortref{Handler}{StmtSort}) associated with the \grammar{try-block}.
\end{itemize}

\note{this representation is subject to change in future versions}

\partition{stmt.try}

\subsection{\valueTag{StmtSort::If}}
\label{sec:ifc:StmtSort:If}

A \type{StmtIndex} abstract reference with tag \valueTag{StmtSort::If} designates a \grammar{selection-statement} of \code{if} sort.
The \field{index} field of that abstract reference is an index into the if-statement partition.
Each entry in that partition is a structure with the following layout
%
\begin{figure}[H]
	\centering
	\structure{
		\DeclareMember{locus}{SourceLocation} \\
		\DeclareMember{initialization}{StmtIndex} \\
		\DeclareMember{condition}{StmtIndex} \\
		\DeclareMember{consequence}{StmtIndex} \\
		\DeclareMember{alternative}{StmtIndex} \\
	}
	\caption{Structure of an if-statement}
	\label{fig:ifc-if-stmt-structure}
\end{figure}
%
The fields have the following meanings:
\begin{itemize}
	\item \field{locus} designates the source location of this \code{if}-statement.
	\item \field{initialization}, if not null, designates an instance of \grammar{init-statement} in the input source.
	\item \field{condition} designates the condition expression of declaration controling this \code{if}-statement.
	\item \field{consequence} designates the branch of this \code{if}-statement to execute when the control condition evaluates to \code{true}.
	\item \field{alternative}, if not null designates the branch of this \code{if}-statement to execute when the control condition evaluates to \code{false}.
\end{itemize}

\partition{stmt.if}


\subsection{\valueTag{StmtSort::For}}
\label{sec:ifc:StmtSort:For}

A \type{StmtIndex} abstract reference with tag \valueTag{StmtSort::For} designates an \grammar{iteration-statement} of \code{for} sort.
The \field{index} field of that abstract reference is an index into the \code{for}-statement partition.
Each entry in that partition is a structure with the following layout
%
\begin{figure}[H]
	\centering
	\structure{
		\DeclareMember{locus}{SourceLocation} \\
		\DeclareMember{initialization}{StmtIndex} \\
		\DeclareMember{condition}{StmtIndex} \\
		\DeclareMember{continuation}{StmtIndex} \\
		\DeclareMember{body}{StmtIndex} \\
	}
	\caption{Structure of a \code{for} \grammar{iteration-statement}}
	\label{fig:ifc-for-stmt-structure}
\end{figure}
%
The fields have the following meanings:
\begin{itemize}
	\item \field{locus} designates the source location of this \code{for} \grammar{iteration-statement}.
	\item \field{initialization}, if not null, designates the \grammar{init-statement} of this \code{for} loop.
	\item \field{condition}, if not null, designates the controling condition of this \code{for} loop.
	\item \field{continuation}, if not null, designates the expression to evaluate after the current iteration of the body of the \code{for} loop is executed.
	\item \field{body} designates the body of this \code{for} loop.
\end{itemize}

\partition{stmt.for}


\subsection{\valueTag{StmtSort::Labeled}}
\label{sec:ifc:StmtSort:Labeled}

A \type{StmtIndex} abstract reference with tag \valueTag{StmtSort::Labeled} designates a \grammar{labeled-statement}.
The \field{index} field of that abstract reference is an index into the \grammar{labeled-statement} partition.
Each entry in that partition is a structure with the following layout
%
\begin{figure}[H]
	\centering
	\structure{
		\DeclareMember{locus}{SourceLocation} \\
		\DeclareMember{type}{TypeIndex} \\
		\DeclareMember{label}{ExprIndex} \\
		\DeclareMember{stmt}{StmtIndex} \\
	}
	\caption{Structure of a \grammar{labeled-statement}}
	\label{fig:ifc-labeled-stmt-structure}
\end{figure}

The fields have the following meanings:
\begin{itemize}
	\item \field{locus} designates the source location of this \grammar{labled-statement}.
	\item \field{type} is unused and will be removed from future version.
	\item \field{label} designates the label of this \grammar{labeled-statement} as follows
		\begin{itemize}
			\item if null, then this is the \code{default} case in a \code{switch-statement}, otherwise
			\item if an integer-constant expression (\sortref{Literal}{ExprSort}), then it is the value of a \code{case} label, otherwise
			\item it must be a identifier used as a label (\sortref{Label}{ExprSort}) to designate the program point of \field{stmt}.
		\end{itemize}
	\item \field{stmt} designates the statement the program point of which is identified by \field{label}.
\end{itemize}

\partition{stmt.labeled}

\subsection{\valueTag{StmtSort::While}}
\label{sec:ifc:StmtSort:While}

A \type{StmtIndex} abstract reference with tag \valueTag{StmtSort::While} designates an \grammar{iteration-statement} of \code{while} sort.
The \field{index} field of that abstract reference is an index into the \code{while}\grammar{-statement} partition.
Each entry in that partition is a structure with the following layout
%
\begin{figure}[H]
	\centering
	\structure{
		\DeclareMember{locus}{SourceLocation} \\
		\DeclareMember{condition}{StmtIndex} \\
		\DeclareMember{body}{StmtIndex} \\
	}
	\caption{Structure of a \code{while}\grammar{-statement}}
	\label{fig:ifc-while-stmt-structure}
\end{figure}
%
The fields have the following meanings:
\begin{itemize}
	\item \field{locus} designates the source location of this \code{while}\grammar{-statement}
	\item \field{condition} designates the condition controling this \grammar{iteration-statement}
	\item \field{body} designates the body of this \grammar{iteration-statement}
\end{itemize}

\partition{stmt.while}


\subsection{\valueTag{StmtSort::Block}}
\label{sec:ifc:StmtSort:Block}

A \type{StmtIndex} abstract reference with tag \valueTag{StmtSort::Block} designates an \grammar{compound-statement}.
The \field{index} field of that abstract reference is an index into the block-statement partition.
Each entry in that partition is a structure with the following layout
%
\begin{figure}[H]
	\centering
	\structure{
		\DeclareMember{locus}{SourceLocation} \\
		\DeclareMember{start}{Index} \\
		\DeclareMember{cardinality}{Cardinality} \\
	}
	\caption{Structure of a block statement}
	\label{fig:ifc-block-stmt-structure}
\end{figure}

The fields have the following meanings:
\begin{itemize}
	\item \field{locus} designates the source location of this block statement
	\item \field{start} designates the slot (in the statement heap partition (\secref{sec:ifc-stmt-heap})) of the first \type{StmtIndex} value in this block statement
	\item \field{cardinality} denotes the number of \type{StmtIndex} values in this bock statement 
\end{itemize}

\note{this representation is subject to change in future versions}


\partition{stmt.block}


\subsection{\valueTag{StmtSort::Break}}
\label{sec:ifc:StmtSort:Break}

A \type{StmtIndex} abstract reference with tag \valueTag{StmtSort::Break} designates an \grammar{jump-statement} of \code{break} sort.
The \field{index} field of that abstract reference is an index into the \code{break}\grammar{-statement} partition.
Each entry in that partition is a structure with the following layout
%
\begin{figure}[H]
	\centering
	\structure{
		\DeclareMember{locus}{SourceLocation} \\
	}
	\caption{Structure of a \code{break} statement}
	\label{fig:ifc-break-stmt-structure}
\end{figure}
%
The field has the following meaning:
\begin{itemize}
	\item \field{locus} designates the source location of this \code{break}\grammar{-statement}
\end{itemize}

\partition{stmt.break}


\subsection{\valueTag{StmtSort::Switch}}
\label{sec:ifc:StmtSort:Switch}

A \type{StmtIndex} abstract reference with tag \valueTag{StmtSort::Switch} designates an \grammar{selection-statement} of \code{switch} sort.
The \field{index} field of that abstract reference is an index into the \code{switch}\grammar{-statement} partition.
Each entry in that partition is a structure with the following layout
%
\begin{figure}[H]
	\centering
	\structure{
		\DeclareMember{locus}{SourceLocation} \\
		\DeclareMember{initialization}{StmtIndex} \\
		\DeclareMember{condition}{ExprIndex} \\
		\DeclareMember{body}{StmtIndex} \\
	}
	\caption{Structure of a switch-statement}
	\label{fig:ifc-switch-stmt-structure}
\end{figure}

The fields have the following meanings:
\begin{itemize}
	\item \field{locus} designates the source location of this \code{switch}\grammar{-statement}
	\item \field{initialization} designates the \grammar{init-statement}, if any, in this \code{switch}\grammar{-statement}
	\item \field{condition} designates the \grammar{condition} scrutinized in this \code{switch}\grammar{-statement}
	\item \field{body} designates the body of this \code{switch}\grammar{-statement}
\end{itemize}

\partition{stmt.switch}

\subsection{\valueTag{StmtSort::DoWhile}}
\label{sec:ifc:StmtSort:DoWhile}

A \type{StmtIndex} abstract reference with tag \valueTag{StmtSort::DoWhile} designates an \grammar{iteration-statement} of \code{do} sort.
The \field{index} field of that abstract reference is an index into the \code{do}\grammar{-statement} partition.
Each entry in that partition is a structure with the following layout
%
\begin{figure}[H]
	\centering
	\structure{
		\DeclareMember{locus}{SourceLocation} \\
		\DeclareMember{condition}{ExprIndex} \\
		\DeclareMember{body}{StmtIndex} \\
	}
	\caption{Structure of a \code{do}-statement}
	\label{fig:ifc-do-stmt-structure}
\end{figure}
%
The fields have the following meanings:
\begin{itemize}
	\item \field{locus} designates the source location of this \code{do}\grammar{-statement}
	\item \field{condition} is the condition controling this \grammar{iteration-statement}
	\item \field{body} is the main statement iterated over by this \grammar{iteration-statement}
\end{itemize}

\partition{stmt.do-while}


\subsection{\valueTag{StmtSort::Goto}}
\label{sec:ifc:StmtSort:Goto}

A \type{StmtIndex} abstract reference with tag \valueTag{StmtSort::Goto} designates an \grammar{jump-statement} of \code{goto} sort.
The \field{index} field of that abstract reference is an index into the \code{goto}\grammar{-statement} partition.
Each entry in that partition is a structure with the following layout
%
\begin{figure}[H]
	\centering
	\structure{
		\DeclareMember{locus}{SourceLocation} \\
		\DeclareMember{target}{ExprIndex} \\
	}
	\caption{Structure of a \code{goto} statement}
	\label{fig:ifc-goto-stmt-structure}
\end{figure}
%
The fields have the following meanings:
\begin{itemize}
	\item \field{locus} designates the source location of this \code{goto}\grammar{-statement}
	\item \field{target} designates the program point (\sortref{Label}{ExprSort}) target of control tranfer
\end{itemize}

\partition{stmt.goto}

\subsection{\valueTag{StmtSort::Continue}}
\label{sec:ifc:StmtSort:Continue}

A \type{StmtIndex} abstract reference with tag \valueTag{StmtSort::Continue} designates an \grammar{jump-statement} of \code{continue} sort.
The \field{index} field of that abstract reference is an index into the \code{continue}\grammar{-statement} partition.
Each entry in that partition is a structure with the following layout
%
\begin{figure}[H]
	\centering
	\structure{
		\DeclareMember{locus}{SourceLocation} \\
	}
	\caption{Structure of a \code{continue} statement}
	\label{fig:ifc-continue-stmt-structure}
\end{figure}
%
The field has the following meaning:
\begin{itemize}
	\item \field{locus} designates the source location of this \code{continue}\grammar{-statement}
\end{itemize}

\partition{stmt.continue}

\subsection{\valueTag{StmtSort::Expr}}
\label{sec:ifc:StmtSort:Expr}

A \type{StmtIndex} abstract reference with tag \valueTag{StmtSort::Expr} designates an \grammar{expression-statement}.
The \field{index} field of that abstract reference is an index into the exression statement partition.
Each entry in that partition is a structure with the following layout
%
\begin{figure}[H]
	\centering
	\structure{
		\DeclareMember{locus}{SourceLocation} \\
		\DeclareMember{expr}{ExprIndex} \\
	}
	\caption{Structure of an expression-statement}
	\label{fig:ifc-expr-stmt-structure}
\end{figure}
%
The fields have the following meanings:
\begin{itemize}
	\item \field{locus} designates the source location of this \grammar{expression-statement}
	\item \field{expr} designates the expression in this \grammar{expression-statement}
\end{itemize}

\partition{stmt.expression}

\subsection{\valueTag{StmtSort::Return}}
\label{sec:ifc:StmtSort:Return}

A \type{StmtIndex} abstract reference with tag \valueTag{StmtSort::Return} designates an \grammar{jump-statement} of \code{return} sort.
The \field{index} field of that abstract reference is an index into the \code{return}\grammar{-statement} partition.
Each entry in that partition is a structure with the following layout
%
\begin{figure}[H]
	\centering
	\structure{
		\DeclareMember{locus}{SourceLocation} \\
		\DeclareMember{type}{TypeIndex} \\
		\DeclareMember{expr}{ExprIndex} \\
		\DeclareMember{function\_type}{TypeIndex} \\
	}
	\caption{Structure of a \code{return} statement}
	\label{fig:ifc-return-stmt-structure}
\end{figure}
%
The fields have the following meanings:
\begin{itemize}
	\item \field{locus} designates the source location of this \code{return}\grammar{-statement}
	\item \field{type} designates the overall converted type of the expression operand to \code{return}
	\item \field{expr} designates the expression to be \code{return}ed
	\item \field{function\_type} designates the type of the function that contains this \code{return}-statement
\end{itemize}

\note{this representation is subject to change in future versions}

\partition{stmt.return}

\subsection{\valueTag{StmtSort::Decl}}
\label{sec:ifc:StmtSort:Decl}

A \type{StmtIndex} abstract reference with tag \valueTag{StmtSort::Decl} designates \grammar{declaration-statement}.
The \field{index} field of that abstract reference is an index into the declaration statement partition.
Each entry in that partition is a structure with the following layout
%
\begin{figure}[H]
	\centering
	\structure{
		\DeclareMember{locus}{SourceLocation} \\
		\DeclareMember{decl}{DeclIndex} \\
	}
	\caption{Structure of a declaration statement}
	\label{fig:ifc-decl-stmt-structure}
\end{figure}
%
The fields have the following meanings:
\begin{itemize}
	\item \field{locus} designates the source location of this declaration statement
	\item \field{decl} designates the declaration (\secref{sec:ifc-decls}) in this statement
\end{itemize}

\partition{stmt.decl}

\subsection{\valueTag{StmtSort::Expansion}}
\label{sec:ifc:StmtSort:Expansion}

A \type{StmtIndex} abstract reference with tag \valueTag{StmtSort::Expansion} designates a statement expansion.
The \field{index} field of that abstract reference is an index into the expansion statement partition.
Each entry in that partition is a structure with the following layout
%
\begin{figure}[H]
	\centering
	\structure{
		\DeclareMember{locus}{SourceLocation} \\
		\DeclareMember{operand}{StmtIndex} \\
	}
	\caption{Structure of an expansion statement}
	\label{fig:ifc-expansion-stmt-structure}
\end{figure}
%
The fields have the following meanings:
\begin{itemize}
	\item \field{locus} designates the source location of this declaration statement
	\item \field{operand} designates the statement (presumably containing a pack) to be expanded
\end{itemize}

\partition{stmt.expansion}

\subsection{\valueTag{StmtSort::SyntaxTree}}
\label{sec:ifc:StmtSort:SyntaxTree}

\begin{figure}[H]
	\centering
	TBD
\end{figure}

\partition{stmt.syntax-tree}

\subsection{\valueTag{StmtSort::Handler}}
\label{sec:ifc:StmtSort:Handler}

A \type{StmtIndex} abstract reference with tag \valueTag{StmtSort::Handler} designates a handler in a \code{try}-statement.
The \field{index} field of that abstract reference is an index into the handler partition.
Each entry in that partition is a structure with the following layout
%
\begin{figure}[H]
	\centering
	\structure{
		\DeclareMember{locus}{SourceLocation} \\
		\DeclareMember{exception}{DeclIndex} \\
		\DeclareMember{body}{StmtIndex} \\
	}
	\caption{Structure of an handler statement}
	\label{fig:ifc-handler-stmt-structure}
\end{figure}
%
The fields have the following meanings:
\begin{itemize}
	\item \field{locus} designates the source location of this handler
	\item \field{exception} designates the declaration of the exception parameter to this handler
	\item \field{body} designates the statement to execute when the exception is bound to the parameter
\end{itemize}

\partition{stmt.handler}

\subsection{\valueTag{StmtSort::Tuple}}
\label{sec:ifc:StmtSort:Tuple}

A \type{StmtIndex} abstract reference with tag \valueTag{StmtSort::Tuple} designates a tuple statement, i.e. a sequence of statements treated as a single unit.
The \field{index} field of that abstract reference is an index into the tuple statement partition.
Each entry in that partition is a structure with the following layout
%
\begin{figure}[H]
	\centering
	\structure{
		\DeclareMember{locus}{SourceLocation} \\
		\DeclareMember{start}{Index} \\
		\DeclareMember{cardinality}{Cardinality} \\
	}
	\caption{Structure of a tuple statement}
	\label{fig:ifc-tuple-stmt-structure}
\end{figure}
%
The fields have the following meanings:
\begin{itemize}
	\item \field{locus} designates the source location of this handler
	\item \field{start} designates the slot (in the statement heap partition (\secref{sec:ifc-stmt-heap})) of the first \type{StmtIndex} value in this block statement
	\item \field{cardinality} denotes the number of \type{StmtIndex} values in this bock statement 
\end{itemize}

\partition{stmt.tuple}

\subsection{\valueTag{StmtSort::Directive}}
\label{sec:ifc:StmtSort:Directive}

A \type{StmtIndex} abstract reference with tag \valueTag{StmtSort::Directive} designates a directive (\secref{sec:ifc-directives}) in grammatical statement context.
The \field{index} of that abstract reference is an index into the directive statement partition.
Each entry in that partition is a structure with the following layout
%
\begin{figure}[H]
	\centering
	\structure{
		\DeclareMember{directive}{DirIndex} \\
	}
	\caption{Structure of a directive statement structure}
	\label{fig:ifc:StmtSort:Directive}
\end{figure}
%
with the following meaning for the field
\begin{itemize}
	\item \field{directive} denotes the \type{Directive} contained in this statement
\end{itemize}

\partition{stmt.directive}
