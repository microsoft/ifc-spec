\label{sec:structure}

\section{Overview}
\label{sec:overview}

Compiled module interfaces are persisted in an IFC format.  The general
representation is as follows
\begin{figure}[hbp]
  \centering
  \begin{tikzpicture}
    \matrix (ifc) [matrix of nodes,
            row sep=-\pgflinewidth,
            nodes={rectangle,draw,anchor=west,text width=10em}]
    {
      {File Signature} \\
      {Header} \\
      \textit{Partition 1} \\
      $\vdots$ \\
      \textit{Partition n} \\
	{String Table} \\
      {Table of Contents}\\
    };
	%% Header gives offsets to ToC, String Table, Global Scope.
	\draw [->] (ifc-2-1.west) to [out=180,in=180] (ifc-7-1.north west);
	\draw [->] (ifc-2-1.west) to [out=180,in=180] (ifc-6-1.north west);
	\draw [->] (ifc-2-1.west) to [out=180,in=180] (ifc-4-1.west);

	\draw[->] (ifc-7-1.east) to [out=0,in=0] (ifc-3-1.north east);
	\draw[->] (ifc-7-1.east) to [out=0,in=0] (ifc-4-1.north east);
	\draw[dotted,->] (ifc-7-1.east) to [out=0,in=0] (ifc-4-1.east);
	\draw[->] (ifc-7-1.east) to [out=0,in=0] (ifc-5-1.north east);
  \end{tikzpicture}
  \caption{IFC file general structure}
  \label{fig:ifc-file-overview}
\end{figure}

\noindent
An IFC represents the \emph{abstract semantics graph} that is the result
of elaborating declarations in an input source file, e.g. a module interface file,
 or a header unit, or indeed any C++ source file that leads to a 
well-formed translation unit.
Declarations and expressions are designated by \emph{abstract references}.  An
abstract reference is essentially a typed pointer that refers to an index in a
partition.  The specific partition is given by the \field{tag} field of the abstract
reference, and the index is given by the \term{index} field of the reference.  All
abstract references are multiple-byte values with 32-bit precision:
%
\begin{figure}[H]
  \centering
	\begin{BasicAbstractReferenceLayout}{32}
		\bitfield{10}{\field{tag}: \clipType{Sort}{N}}
      		\bitfield{22}{\field{index}: \clipType{Index}{32 - N} }
		\bitFormatTextAt{N-1}{8}
 		\bitSeparate{10}
		\bitFormatTextAt{N}{10}
	\end{BasicAbstractReferenceLayout} 
  \caption{Abstract reference parameterized by the sort of designated entity.}
  \label{fig:ifc-abstract-reference}
\end{figure}
%

\noindent
The overall aim is to define the binary representations after the 
\emph{Internal Program Representation} (IPR), a work done by Gabriel Dos~Reis
and Bjarne Stroustrup \cite{gdr-bs:ipr-macis-special-issue,ipr:web} to define a more regular 
foundational semantics for
C++, capable of capturing ISO C++ and practical dialects.

\section{Multiple IFCs per file}
\label{sec:ifc-multiple-ifcs-per-file}

The current specification only defines one IFC per containing file.  However, the long term goal is to support multiple IFCs per contaning file.
Where there is a mention of "offset from the beginning of the file", it should be understood "offset from the beginning of the current IFC".

\section{Type of IFC container}
\label{sec:ifc-type-of-ifc-container}

An IFC can be embedded in just any binary file.  The VC++ compiler by default generates IFCs in binary files with the 
\code{.ifc} extension.  However, they can also be embedded in archives (e.g. files with \code{.a} or \code{.lib} extensions), or
in shared or dynamically linked libraries (e.g. files with \code{.so}, \code{.dll}, \code{.dylib} extensions.)

\section{On-demand materialization}

The IFC format is designed to support (and encourage) ``on-demand''
materialization of declarations.  That is, when the compiler sees an
import-declaration, it does \emph{not} bring in all the declarations
right away.  Rather, the idea is that it only makes visible
the set of (toplevel) names exported by the nominated module.  An
on-demand materialization strategy will only reconstruct declarations
upon name lookup (in response to a name use in the importing
translation unit) however referenced.  The on-demand materialization
strategy embodies C++'s long standing philosophy of ``you don't pay
for what you don't use.''

\section{Endianness}
\label{sec:ifc-endianness}

Each multibyte scalar value used in the description of an IFC file header (\secref{sec:ifc-file-header})
and in the table of contents (\secref{sec:ifc-toc})
is stored in little endian format. Multibyte scalar values stored in the partitions use the
endianness of the target architecture.

\section{Basic data types}
\label{sec:ifc-basic-data-types}

This document uses a few fundamental data types, \type{u8}, \type{u16}, and \type{32} with the following characteristics:
\begin{itemize}
	\item \type{u8}: $1$ octet, with alignment $1$; usually equivalent to C++'s \code{uint8_t}
	\item \type{u16}: $2$ octets, with alignment $2$; usually equivalent to C++'s \code{uint16_t}
	\item \type{u32}: $4$ octets, with alignment %4%; usually equivalent to C++'s \code{uint32_t} 
\end{itemize} 

\subsection{File offset in bytes}
At various places, the locations of certain tables (especially partitions) are
described in terms of byte offset from the beginning of the current IFC file.  The
document uses the following abstract data type for those quantities.
\newtype{ByteOffset}{32}

\paragraph{Note:}
The current implementation uses 4-byte for file offsets, but that will change
in coming updates to 8-byte in anticipation of large IFC file support.

\subsection{Cardinality: counting items}
At various places, the IFC indicates how many elements there are in a given
table.  That information is given by a 32-bit integer value abstacted as
follows: \newtype{Cardinality}{32}

\subsection{Extent of entities}
\label{sec:ifc-entity-size}
At various places, in partitular in partition summaries (\secref{sec:ifc-partition}), the IFC needs to indicate 
the number of bytes contain in entity representation.  That information is
indicated by a 32-bit value of type \newtype{EntitySize}{32}

\subsection{Generic Indices}
\label{sec:ifc-index-type}
The type of generic indices into tables is defined as \newtype{Index}{32}

\subsection{Sequence}
\label{sec:ifc-sequence}
A sequence is generically described by a pair:
%
\begin{figure}[H]
	\centering
	\structure{
		\DeclareMember{start}{Index} \\
		\DeclareMember{cardinality}{Cardinality}\\
	}
	\caption{Structure of a sequence}
	\label{fig:ifc-sequence-structure}
\end{figure}
%
The \field{start} field is an index into the partition the sequence is part of.  It designates the first item in the sequence.
The \field{cardinality} designates the number of items in the sequence.

\subsection{Content Hash}
\label{sec:content-sha256-hash}
The interesting portion of the content of an IFC file is hashed using SHA-256 algorithm, and stored as a value of type \type{SHA256}:
A basic data type with 256 bits width, and with alignment 4.

\subsection{File Format Versioning}
\label{sec:ifc-versioning-data-type}

Each IFC header has version information, major and minor of type defined as \newtype{Version}{8}

\subsection{ABI}
\label{sec:ifc-abi-data-type}
The ABI targeted by an IFC is recorded in a field of the IFC header, of type \newtype{Abi}{8}

\subsection{Architecture}
\label{sec:ifc-architecture-data-type}
The architecture targeted by an IFC is recorded in a field of the IFC header, of type
\begin{typedef}{Architecture}{}
	enum class Architecture : uint8_t {
		Unknown	= 0x00,		// Unknown target
		X86			= 0x01,		// x86 (32-bit) target
		X64			= 0x02,		// x64 (64-bit) target
		ARM32		= 0x03,		// ARM (32-bit) target
		ARM64		= 0x04,		// ARM (64-bit) target
		HybridX86ARM64 = 0x05,	// Hybrid x86-arm64
	};
\end{typedef}

\subsection{Language Version}
\label{sec:ifc-language-version}

The C++ language version is a 32-bit value of type \newtype{LanguageVersion}{32}

\section{IFC File Signature}
\label{sec:ifc-file-signature}

Each valid IFC file (see \figref{fig:ifc-file-overview}) starts with the following 4-byte file signature:
\begin{lstlisting}
  0x54  0x51 0x45 0x1A
\end{lstlisting}


\section{IFC File Header}
\label{sec:ifc-file-header}

Following the file signature (\secref{sec:ifc-file-signature}) is a header that describes a checksum, 
format version information, ABI information, target architecture information, C++ language version,
 the offset to the string table (\secref{sec:ifc-string-table}) and how long it is, 
the name of the IFC's module, the filename of the original C++ source file, the index of the global scope,
then an indication on
where to find the ``table of contents'' (\secref{sec:ifc-toc}), and finally how many partitions (\secref{sec:ifc-partition})
the IFC contains.
%
\begin{figure}[H]
  \centering
    \structure[text width = 15em]{
	\DeclareMember{checksum}{SHA256} \\
	\DeclareMember{major\_version}{Version} \\
      \DeclareMember{minor\_version}{Version} \\
      \DeclareMember{abi}{Abi} \\
      \DeclareMember{arch}{Architecture} \\
	\DeclareMember{dialect}{LanguageVersion} \\
	\DeclareMember{string\_table\_bytes}{ByteOffset} \\
	\DeclareMember{string\_table\_size}{Cardinality} \\
	\DeclareMember{unit}{UnitIndex} \\
	\DeclareMember{src\_path}{TextOffset} \\
	\DeclareMember{global\_scope}{ScopeIndex} \\
      \DeclareMember{toc}{ByteOffset} \\
	\DeclareMember{partition\_count}{Cardinality} \\
	\DeclareMember{internal}{bool} \\
    }
  \caption{Structure of an IFC binary file header}
  \label{fig:ifc-file-header}
\end{figure}

The interpretation of the fields is as follows:
\begin{description}
\item[Content checksum]
	The field \field{checksum} represents the SHA-256 hash of the portion of the IFC file content starting from right after 
	that field to the end of the IFC file.

\item[Version]
	The fields \field{major\_version} and \field{minor\_version} collectively denote the version of the data structures in the
	IFC. The current format version is $0.25$, meaning \field{major\_version} is $0$,
	and \field{minor\_version} is $25$.

\item[Target ABI] 
	The field \field{abi} records the ABI of the target platform of the IFC.

\item[Target Architecture]
	The field \field{arch} records the architecture targeted by the IFC.

\item[C++ Language Version]
	The field \field{dialect} records the value of the C++ pre-defined macro \code{\_\_cplusplus} in effect
	when the IFC was created.

\item[String Table]
	The field \field{string\_table\_bytes} is an offset from the beginning of the IFC to the first byte of the string table (\secref{sec:ifc-string-table}).
	the number of bytes in table is indicated by \field{string\_table\_size}.

	The string table holds the representation of any single strings or identifiers in the IFC.  Consequently, locating it is
	essential for determining the IFC's module name, and also for locating partitions by name.

\item[Translation Unit Descriptor]
	A classification of the translation unit for which this IFC was generated is described by the field \field{unit},  value of
     type \type{UnitIndex} (\secref{sec:ifc-tu}).

\item[Source Pathname]
	The filename of the C++ source file out of which the IFC was produced is indicated by the field \field{src\_path}, an
	offset into the string table.  In the current implementation, this is an ordinary NUL-terminated narrow string.

\item[Global Scope]
	Every declaration is rooted in the global namespace.  The field \field{global\_scope} is index into the
	scope partition (\secref{sec:ifc-scope-desc}), pointing to the description of the global namespace.  Traversing the global scope, and recursively any 
	contained declaration, gives the entire abstract semantics graph making up the IFC.

\item[Table of Contents]
	The table of contents is an array of all partition summaries in the IFC.  The field \field{toc} indicates the offset (in bytes)
	from the beginning of the offset to the first partition descriptor.  The number of partition summaries in the table of contents
	is given by the field \field{partition\_count}.

\item[Internal Unit]
	Whether the IFC is for an exported module unit or not is indicated by \field{internal}.  This field is false for all
	translation units are produced except non-exported module partitions.

\end{description}

\paragraph{Note}
The values of the major and minor versions, the ABI, and the architecture fields
are not fixed yet.   All multi-byte integer values in header are stored according to a little endian format.  All multi-byte integer values
stored in the partitions are stored according to the endianness of the target architecture.
The structure of the field {internal} may change in future revisions.


%% \paragraph{ToC offset}
%% The offset to the table of contents is currently
%% encoded over 4 bytes, but that may change soon to 8 bytes.  
%% This field
%% indicates the relative position (in bytes) of the \code{TableOfContents} from
%% the beginning of the IFC file.

\section{IFC Table of Contents}
\label{sec:ifc-toc}

The data in an IFC are essentially homoegenous tables (called
\emph{partitions}) with entries referencing each other.  

The table of 
contents is written near the end of the IFC file as that arrangement
allows one-pass algorithms for writing out IFC files while minimizing
the amount of intermediary internal storage needed to compute the full
abstract semantics graph.

\subsection{Partition}
\label{sec:ifc-partition}

Each partition is described by a \emph{partition summary} information with the
following layout
\begin{figure}[H]
  \centering
  \structure{
      \DeclareMember{name}{TextOffset} \\
      \DeclareMember{offset}{ByteOffset} \\
      \DeclareMember{cardinality}{Cardinality} \\
      \DeclareMember{entry\_size}{EntitySize} \\
    }
  \caption{Partition summary}
  \label{fig:ifc-partition-summary}
\end{figure}

\begin{description}
\item[name] An index into the string pool.  It points to the name (a
  NUL-terminated character string) of the partition.

\item[offset] Location (file offset in bytes) of the partition relative to the
  beginning of the IFC file.

\item[cardinality] The number of items in the partition.

\item[entry\_size] The (common) size of an item in the partition.
\end{description}

So, the byte count of a partition is obtained by multiplying the individual
\field{entry\_size} by the \field{cardinality}.

\section{Elaboration vs. syntax tree}
\label{sec:ifc-elaboration-vs-syntax-tree}

The IFC, like the IPR, is designed to represent all of C++, including extensions.  This means representing faithfully
non-template entities as well as template entities.  An \term{elaboration} of an entity is the result of full semantics
anlysis (e.g. the result of name lookup, type checking, overload resolution, template specialization if needed, etc.) of that
entity.  A node, in the abstract semantics graph of an IFC, representing a non-template is an elaboration.

By contrast, semantics analysis of templates proceeds in two steps, by language definition.  For example, in a template code where \code{T} is a type parameter, the meaning of the expression \code{T\{ 42 \}} depends both on the meaning and structure of the 
actual argument value for \code{T}.  It could be a constructor invocation, or a conversion function call, or a non-narrowing
static cast.  Consequently, representations of templates need to be fairly syntactic since only instantiations are fully semantically
analyzed.  Syntax trees (\secref{sec:ifc-syntax-tree-table}) are used to represent templates. 
That representation occasionally contains nodes that are
elaborations, since a certain amount of semantics analysis is required when parsing template definitions.

