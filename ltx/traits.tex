\label{sec:ifc-traits}

A trait is a property of an entity not directly stored in the data structure representing that entity.
Traits are stored in associative table partitions.  Examples of traits include the deprecation 
message associated (\secref{sec:ifc-deprecated-trait}) with a declaration,
friend declarations (\secref{sec:ifc-friend-trait}), the set of specializations of a 
template (\secref{sec:ifc-specialization-trait}), etc.
 Each entry in a (parameterized) trait table is a pair structure of the following form
\begin{figure}[H]
	\centering
	\structure{
		\DeclareMember{decl}{DeclIndex} \\
		\DeclareMember{trait}{T} \\
	}
	\caption{Structure of \type{AssociatedTrait<T>}}
	\label{fig:ifc-associated-trait-structure}
\end{figure}
%
where the \field{decl} field is an abstract reference designating the entity, and the \field{trait} designates the property associated with the entity.
The entries in a trait table are stored by increasing values of the \field{decl} field.  There is at most one entry for a given \field{decl} per trait
table.  Note however that some \field{trait} may be tuples of sort \type{T} when there are conceptually multiple traits associated with a \field{decl}.

\section{Deprecation texts}
\label{sec:ifc-deprecated-trait}

The type \type{T} of the \field{trait} associated with \field{decl} is \type{TextOffset}, designating
the string literal of the deprecation message.

\partition{trait.deprecated}

\note{This representation is subject to change.}


\section{Template specializations}
\label{sec:ifc-specialization-trait}

For each template \field{decl}, the set of declarations for corresponding partial or explicit specializations
is stored as a sequence in the scope member partition, and the description of that sequence is given by the template specialization trait.


The type \type{T} of the \field{trait} associated with \field{decl} is \type{Sequence} (\secref{sec:ifc-sequence})
of \type{Declaration}s (\figref{fig:ifc-declaration-structure}),
denoting the sequence of descriptions for each specialization of the template \field{decl}.  The
value of the field \field{trait.start} is an index into the scope member partition 
(\secref{sec:ifc-scope-member}), and the field \field{trait.cardinality} gives the number of \type{Declaration}s
in that sequence.

\partition{trait.specialization}

\note{This representation is subject to change.}

\section{Friendship of a class}
\label{sec:ifc-friend-trait}

For each declaration \field{decl} of a class type, the set of declared friends of that class type 
is stored as a sequence in the scope member partition, and the description of that sequence is given by the friend trait.


The type \type{T} of the \field{trait} associated with \field{decl} is \type{Sequence} (\secref{sec:ifc-sequence})
of \type{Declaration}s (\figref{fig:ifc-declaration-structure}),
denoting the sequence of descriptions for each declared friend of \field{decl}.  The
value of the field \field{trait.start} is an index into the scope member partition 
(\secref{sec:ifc-scope-member}), and the field \field{trait.cardinality} gives the number of \type{Declaration}s
in that sequence.

\partition{trait.friend}

\note{This representation is subject to change.}

\section{Function Definition}
\label{sec:ifc-function-definition-trait}

The trait of a definition of a constexpr function or constructor is represented by a structure of defined as follows
\begin{figure}[H]
	\centering
	\structure{
		\DeclareMember{parameters}{ChartIndex} \\
		\DeclareMember{initializers}{ExprIndex} \\
		\DeclareMember{body}{StmtIndex} \\
	}
	\caption{Structure of function definition}
	\label{fig:ifc-function-definition-structure}
\end{figure}
%
The \field{parameters} field designates the chart of parameters (\secref{sec:ifc-charts}) of this function or constructor.
The \field{initializers} field designates the set of member-initializers, if any, of the constexpr constructor.
The \field{body} field designates the statement body (\secref{sec:ifc-statements}) of the function or constructor.

\note{This structure is subject to change.}

\partition{trait.mapping-expr}

\section{Template Alias}
\label{sec:ifc-template-alias-trait}

MSVC, in its current form, uses syntax trees (\secref{sec:ifc-syntax-tree-table}) to represent
declarations of template aliases.  For each template alias declaration \field{decl}, the type \type{T}
of the corresponding \field{trait} is \type{SyntaxIndex} (\figref{fig:ifc-syntax-index}).

\partition{trait.alias-template}

\note{This trait is subject of removal in future releases.}

\section{Declared deduction guides}
\label{sec:ifc:deduction-guides-trait}

For each class template \field{decl} with declared deduction guides, 
the type \type{T} of the corresponding \field{trait} is \type{DeclIndex} (\figref{fig:ifc-decl-index}).  The sort of \field{trait}
is \valueTag{DeclSort::Tuple} (\sortref{Tuple}{DeclSort}) when the number of deduction guides is greater than one.


\partition{trait.deduction-guides}

\section{Trailing constraints}
\label{sec:ifc:trailing-constraints}

For each templated function declaration \field{decl} with trailing \grammar{requires-clause} ---
hence having a \code{FunctionTraits::Constrained} (\secref{sec:ifc-function-traits}) --- the 
associated constraint condition is stored 
(in syntax tree form) in the trailing constraints trait partition.  The type \type{T} of \field{trait} is
\type{SyntaxIndex} (\figref{fig:ifc-syntax-index}).

\partition{trait.requires}

\note{This structure is subject to change}.

\section{Declaration attributes}
\label{sec:ifc:decl-attributes}

Each declaration \field{decl} with attributes not understood in the form of 
\type{ObjectTraits} or \type{FunctionTraits} -- or more generally attributes of templated declarations --
has an entry in the declaration attribute partition.  The type \type{T} of \field{trait} is 
\type{AttrIndex} (\figref{fig:ifc-attr-index}).

\partition{trait.attribute}

\note{This representation is subject to change.}

\section{MSVC vendor-specific traits}
\label{sec:ifc-msvc-vendor-specific-trait}

The MSVC compiler associates a certain set of vendor-specific traits to most declarations.  
These vendor-specific traits are represented by the 32-bit bitmask type \type{MsvcTraits}
defined in \secref{sec:ifc-msvc-trait-bitset}.

\partition{.msvc.trait.vendor-traits}


\subsection{Trait for MSVC UUID}
\label{sec:ifc-msvc-uuid-trait}

The MSVC compiler allows source-level association of UUID with any non-local declaration. When that occurs,
the associated MSVC vendor-specific trait has the \valueTag{MsvcTraits::Uuid} bit set.  The  UUID is a
16-byte integer in the MSVC UUID trait table.

\partition{.msvc.trait.uuid}

\subsection{Function parameters in functions definitions}
\label{sec:ifc-msvc-fun-parms}

There is an infelicity in the current MSVC implementation where function parameter names are available only in defining function declarations.
The corresponding parameter names, and associated default arguments, are stored in a trait associated with the declaration, see \figref{fig:ifc-associated-trait-structure}.
The type of the \field{trait} field
is \type{ChartIndex} (\secref{sec:ifc:ChartSort:Unilevel}) listing the sequence of parameter declarations -- name, type, and default argument (if any) -- in that definition.

\partition{.msvc.trait.named-function-parameters}


\subsection{Attributes on declarations}
\label{sec:ifc-msvc-decl-attr-trait}

The MSVC compiler records an attribute on a declaration in this trait table.

\partition{.msvc.trait.decl-attrs}

\note{This trait is scheduled for removal in future releases of MSVC.}


\subsection{Attributes on Statements}
\label{sec:ifc-msvc-stmt-attr-trait}

The MSVC compiler records an attribute on a statement in this trait table.

\partition{.msvc.trait.stmt-attrs}

\note{This trait is scheduled for removal in future releases of MSVC.}


\subsection{Mapping expressions for code generation}
\label{sec:ifc-msvc-codegen-mapping-trait}

The MSVC compiler currently has an infelicity in the representations for code generation of runtime behavior.
This trait associates the runtime codegen for inline functions.

\partition{.msvc.trait.codegen-mapping-expr}

\note{This trait is scheduled for removal in future releases of MSVC.}


\subsection{Dynamic initialization of variables}
\label{sec:ifc-msvc-dynamic-init-trait}

The MSVC implementation of dynamic initialization of variables with static duration needing non-trivial initialization uses a synthesized linker-level symbol.
This trait associates such a source-level variable with the corresponding linker-level symbol.

\partition{.msvc.trait.dynamic-init-variable}

\subsection{Code generation properties of labels}
\label{sec:ifc-msvc-codegen-labels-properties-trait}

The MSVC coompiler needs to record some properties of labels for code generation purposes.
This trait maps a label to the associated code generation properties.

\partition{.msvc.trait.codegen-label-properties}

\note{This trait is scheduled for removal in future releases of MSVC.}


\subsection{Code generation type view of \code{switch} statement}
\label{sec:ifc-msvc-codegen-switch-type-trait}

The MSVC compiler exhibits this oddity where it needs to associate a type with the labels and condition in a \code{switch}-statetement.
This trait associate a \code{switch} statement with such a type.

\partition{.msvc.trait.codegen-switch-type}

\note{This trait is scheduled for removal in future releases of MSVC.}

\subsection{Code generation trait for \code{do}-statement}
\label{sec:ifc-msvc-codegen-do-stmt-trait}

The MSVC compiler has a weakness in its codegeneration representation of \code{do}-statement.  This traits associates
additional expressions needed for runtime code generation with a \code{do}-statement.

\partition{.msvc.trait.codegen-dowhile-stmt}

\note{This trait is scheduled for removal in future releases of MSVC.}

\subsection{Lexical scope of a declaration}
\label{sec:ifc-msvc-lexical-scope-trait}

The MSVC compiler has a weakness where it needs to known the lexical scope of a given declaration in order to compute the appropriate linker-level name decoration.
This trait associates a declaration with its lexical scope (when different from its home scope).

\partition{.msvc.trait.lexical-scope-index}

\note{This trait is scheduled for removal in future releases of MSVC.}

\subsection{Range of source file line numbers}
\label{sec:ifc-msvc-source-file-line-range-trait}

This trait associates a source file with a pair minimum and maximum line numbers in that source file.

\partition{.msvc.trait.file-boundary}

\note{This trait is scheduled for removal in future releases of MSVC.}

\subsection{Source File of a header unit}
\label{sec:ifc-msvc-header-unit-source-file-trait}

The MSVC associates a header name (as it appears in the source code) with the source file it ultimately resolves to.
This information should be made standard in the normal IFC representation.

\partition{.msvc.trait.header-unit-source-file}

\note{This trait is scheduled for removal in future releases of MSVC.}
