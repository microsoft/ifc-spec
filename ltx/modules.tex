\label{sec:ifc-modules}

Any translation unit can import any module.  Additionally, a module interface can re-export an imported module.

\section{Module description structures}
\label{sec:ifc:module-structures}

\subsection{Module reference}
\label{sec:ifc--module-reference}

All used modules (whether imported or exported) are represented as module references of type defined as follows
%
\begin{figure}[h]
	\centering
	\structure{
		\DeclareMember{owner}{TextOffset} \\
		\DeclareMember{partition}{TextOffset} \\
	}
	\caption{Structure of a \type{ModuleReference}}
	\label{fig:ifc-module-reference-structure}
\end{figure}
%

The fields of a module references have the following meanings:
\begin{itemize}
	\item[owner] This value designates the name of the module.  A null name indicated the global module.

	\item[partition] This value designates the partition of the owning module.  When the partition name is null, the reference is to
	the primary module interface, otherwise it designates the partition of the owning module.  When the owner is the global module
	then the partition designates the source file representing that partition of the global module.
\end{itemize}


\subsection{Imported modules}
\label{sec:ifc-imported-module}

References to all imported modules (which are not also exported) are stored in the imported modules partition.

\partition{module.imported}


\subsection{Exported modules}
\label{sec:ifc-exported-module}

References to all exported modules are stored in the exported modules partition.

\partition{module.exported}
