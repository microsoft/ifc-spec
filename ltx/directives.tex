\label{sec:ifc-directives}

Certain ISO C++ grammatical constructs, such as \grammar{asm-declaration}s or \grammar{using-directive}s, are classified as \grammar{declaration}s but they declare no names.
Rather, their elaboration effects are to modify the behavior of the C++ translator.  They are, in effect, directives -- i.e. constructs that specify how a C++ 
translator should process the input source program.

This document uses \type{DirIndex} to indicate a typed abstract reference to a directive.
Like any abstract reference, it is a $32$-bit value
\begin{figure}[htbp]
    \centering
    \absref{5}{DirSort}
    \caption{\type{DirIndex}: Abstract reference to a Directive}
    \label{fig:ifc-dir-index}
\end{figure}

\begin{SortEnum}{DirSort}
    \enumerator{VendorExtension}
    \enumerator{Empty}
    \enumerator{Attribute}
    \enumerator{Pragma}
    \enumerator{Using}
    \enumerator{DeclUse}
    \enumerator{Expr}
    \enumerator{StructuredBinding}
    \enumerator{SpecifiersSpread}
    \enumerator{Unused0}
    \enumerator{Unused1}
    \enumerator{Unused2}
    \enumerator{Unused3}
    \enumerator{Unused4}
    \enumerator{Unused5}
    \enumerator{Unused6}

    \enumerator{Unused7}
    \enumerator{Unused8}
    \enumerator{Unused9}
    \enumerator{Unused10}
    \enumerator{Unused11}
    \enumerator{Unused12}
    \enumerator{Unused13}
    \enumerator{Unused14}
    \enumerator{Unused15}
    \enumerator{Unused16}
    \enumerator{Unused17}
    \enumerator{Unused18}
    \enumerator{Unused19}
    \enumerator{Unused20}
    \enumerator{Unused21}
    \enumerator{Tuple}
\end{SortEnum}

\note{The values of the various \emph{UnusedNN} tag are subject to change in future releases}

\section{Directive Vocabulary Types}
\label{sec:ifc-dir-support-types}

\subsection{Phases of Translation}
\label{sec:ifc-translation-phases-type}

A given directive is evaluated at certain translation phases, specified as a bitmask value of type
%
\begin{typedef}{Phases}{}
    enum class Phases : uint32_t {
        Unknown         = 0,
        Reading         = 1 << 0,
        Lexing          = 1 << 1,
        Preprocessing   = 1 << 2,
        Parsing         = 1 << 3,
        Name_resolution = 1 << 4,
        Typing          = 1 << 5,
        Evaluation      = 1 << 6,
        Instantiation   = 1 << 7,
        Code_generation = 1 << 8,
        Linking         = 1 << 9,
        Loading         = 1 << 10,
        Execution       = 1 << 11,
    };
\end{typedef}
%

\section{Directive Structures}
\label{sec:ifc-dir-structures}

\subsection{\valueTag{DirSort::VendorExtension}}
\label{sec:ifc:DirSort:VendorExtension}

A \type{DirIndex} value with tag \valueTag{DirSort::VendorExtension} represents an abstract reference to a vendor-specific directive.
This tag value is reserved for encoding vendor-specific extensions.
  
\partition{dir.vendor-extension}

\subsection{\valueTag{DirSort::Empty}}
\label{sec:ifc:DirSort:Empty}

A \type{DirIndex} value with tag \valueTag{DirSort::Empty} represents an abstract reference to an \grammar{empty-declaration}.
The \field{index} field is an index into the empty declaration partition.
Each entry in that partition is a structure with the following layout
%
\begin{figure}[H]
    \centering
    \structure{
        \DeclareMember{locus}{SourceLocation} \\
    }
    \caption{Structure of an \grammar{empty-declaration}}
    \label{fig:ifc:DirSort:Empty}
\end{figure}
%
with the following meaning of the field
\begin{itemize}
    \item \field{locus} denotes the source location of this \grammar{empty-declaration}, which is also the source location of the sole semi-colon in the declaration
\end{itemize}

\partition{dir.empty}

\note{At block scope, an \sortref{Empty}{DirSort} may be grammatically undistinguishable from an \grammar{empty-statement}.  
    The C++ translator is expected to construct the appropriate structure based on the context of processing.}


\subsection{\valueTag{DirSort::Attribute}}
\label{sec:ifc:DirSort:Attribute}

A \type{DirIndex} value with tag \valueTag{DirSort::Attribute} represents an abstract reference to an \grammar{attribute-declaration}.
The \field{index} field is an index into the attribute declaration partition.
Each entry in that partition is a structure with the following layout
%
\begin{figure}[H]
    \centering
    \structure{
        \DeclareMember{locus}{SourceLocation} \\
        \DeclareMember{attr}{AttrIndex} \\
    }
    \caption{Structure of an \grammar{attribute-declaration}}
    \label{fig:ifc:DirSort:Attribute}
\end{figure}
%
with the following meanings for the fields
\begin{itemize} 
    \item \field{locus} denotes the source location of this \grammar{attribute-declaration}
    \item \field{attr} denotes the attribute (\secref{sec:ifc-attrs}) in this \grammar{attribute-declaration}
\end{itemize}

\partition{dir.attribute}

\note{At block scope, an \sortref{Attribute}{DirSort} may be grammatically undistinguishable from an attributed \grammar{empty-statement}.  
    The C++ translator is expected to construct the appropriate structure based on the context of processing.}


\subsection{\valueTag{DirSort::Pragma}}
\label{sec:ifc:DirSort:Pragma}

A \type{DirIndex} value with tag \valueTag{DirSort::Pragma} represents an abstract reference to a pragma directive.
The \field{index} field is an index into the pragma directive partition.
Each entry in that partition is a structure with the following layout
%
\begin{figure}[H]
    \centering
    \structure{
        \DeclareMember{locus}{SourceLocation} \\
        \DeclareMember{words}{SentenceIndex} \\
    }
    \caption{Structure of a pragma directive}
    \label{fig:ifc:DirSort:Pragma}
\end{figure}
%
with the following meanings for the fields
\begin{itemize}
    \item \field{locus} denotes the source location of the entire pragma directive
    \item \field{words} denotes the sentence index (\secref{sec:ifc-sentence}) of the words (\secref{sec:ifc-words}) making up the directive
\end{itemize}

\partition{dir.pragma}

\subsection{\valueTag{DirSort::Using}}
\label{sec:ifc:DirSort:Using}

A \type{DirIndex} value with tag \valueTag{DirSort::Using} represents an abstract reference to a \grammar{using-directive} declaration.
The \field{index} field is an index into the using-directive partition.
Each entry in that partition is a structure with the following layout
%
\begin{figure}[H]
    \centering
    \structure{
        \DeclareMember{locus}{SourceLocation} \\
        \DeclareMember{nominated}{ExprIndex} \\
        \DeclareMember{resolution}{DeclIndex} \\
    }
    \caption{Structure of a \grammar{using-directive}}
    \label{fig:ifc:DirSort:Using}
\end{figure}
%
with the following meanings for the fields
\begin{itemize}
    \item \field{locus} denotes the source locatio of this \grammar{using-directive}
    \item \field{nominated} is a \grammar{qualified-id} (\sortref{QualifiedName}{ExprSort}) that designates a namespace, as written in the input source program
    \item \field{resolution} denotes the namespace computed based on the elaboration rules of \grammar{using-directive}.  
    It is a \sortref{Scope}{DeclSort} abstract reference that denotes either
      the namespace designated by \field{nominated} or an enclosing ancestor of the scope \field{nominated}.
\end{itemize}
  
\partition{dir.using}


\subsection{\valueTag{DirSort::DeclUse}}
\label{sec:ifc:DirSort:DeclUse}

A \type{DirIndex} value with tag \valueTag{DirSort::DeclUse} represents an abstract reference to a \grammar{using-declaration} declaration.
The \field{index} field is an index into the using-declaration partition.
Each entry in that partition is a structure with the following layout
%
\begin{figure}[H]
    \centering
    \structure{
        \DeclareMember{locus}{SourceLocation} \\
        \DeclareMember{path}{ExprIndex} \\
        \DeclareMember{result}{DeclIndex} \\
    }
    \caption{Structure of a \grammar{using-declaration}}
    \label{fig:ifc:DirSort:DeclUse}
\end{figure}
%
with the following meanings for the fields
\begin{itemize}
    \item \field{locus} denotes the source location of this \grammar{using-declaration}
    \item \field{path} is a path expression to the declaration(s) target of this \grammar{using-declaration}.  Note that \field{path} can be a normal \sortref{QualifiedName}{ExprSort} or an \sortref{Expansion}{ExprSort} of a \sortref{QualifiedName}{ExprSort}.
    \item \field{result} denotes the declaration(s) resulting from executing this \grammar{using-declaration}.  A null value indicates that the \field{path} has not been resolved yet.
\end{itemize}

\partition{dir.decl-use}


\subsection{\valueTag{DirSort::Expr}}
\label{sec:ifc:DirSort:Expr}

A \type{DirIndex} value with tag \valueTag{DirSort::Expr} represents an abstract reference to a phased evaluation of an expression.
The \field{index} field is an index into the phased evaluation partition.
Each entry in that partition is a structure with the following layout
%
\begin{figure}[H]
    \centering
    \structure{
        \DeclareMember{locus}{SourceLocation} \\
        \DeclareMember{expr}{ExprIndex} \\
        \DeclareMember{phases}{Phases} \\
    }
    \caption{Structure of a phased evaluation directive}
    \label{fig:ifc:DirSort:Expr}
\end{figure}
%
with the following meanings of the fields
\begin{itemize}
    \item \field{locus} denotes the source location of this phased evaluation directive
    \item \field{expr} denotes the expression (\secref{sec:ifc-exprs}) to evaluate
    \item \field{phases} denotes the set of phases of translation at this which this directive is to be evaluated
\end{itemize}
  
\partition{dir.expr}


\subsection{\valueTag{DirSort::StructuredBinding}}
\label{sec:ifc:DirSort:StructuredBinding}

A \type{DirIndex} value with tag \valueTag{DirSort::StructuredBinding} represents an abstract reference to a structured binding declaration.
The \field{index} field is an index into the structure binding declaration partition.
Each entry in that partition is a structure with the following layout
%
\begin{figure}[H]
    \centering
    \structure{
        \DeclareMember{locus}{SourceLocation} \\
        \DeclareMember{bindings}{DeclSequence} \\
        \DeclareMember{specifiers}{DeclSpecifierSequence} \\
        \DeclareMember{names}{IdentifierSequence} \\
        \DeclareMember{initializer}{ExprIndex} \\
        \DeclareMember{ref}{BindingMode} \\
    }
    \caption{Structure of a structured binding declaration directive}
    \label{fig:ifc:DirSort:StructuredBinding}
\end{figure}
%
with the following meanings for the fields
\begin{itemize}
    \item \field{locus} denotes the source location of this structured binding declaration
    \item \field{bindings} is an index into the \sortref{Tuple}{DeclSort} partition denoting the sequence of declarations resulting from the elaboration of this structured binding declaration
    \item \field{specifiers} denotes the sequence of \grammar{decl-specifier}s in this declaration
    \item \field{names} denotes the sequence of identifiers in this declaration
    \item \field{ref} denotes the \grammar{ref-qualifier} in this structured binding declaration
\end{itemize}

\partition{dir.struct-binding}


\subsection{\valueTag{DirSort::SpecifiersSpread}}
\label{sec:ifc:DirSort:SpecifiersSpread}

A \type{DirIndex} value with tag \valueTag{DirSort::SpecifiersSpread} represents an abstract reference to a spread of sequence of \grammar{decl-specifier}s 
over a collection of \grammar{init-declarator}s in a single grammatical \grammar{declaration}. 
Those are directives to the C++ translator to synthesize multiple declarations as part of the elaboration of that grammar production.
The \field{index} field of the abstract reference is an index into the specifiers spread partition.
Each entry in that partition is a structure with the following layout
%
\begin{figure}[H]
    \centering
    \structure{
        \DeclareMember{locus}{SourceLocation} \\
        \DeclareMember{specifiers}{DeclSpecifierSequence} \\
        \DeclareMember{targets}{ProclamatorSequence} \\
    }
    \caption{Structire of a specifiers spread directive}
    \label{fig:ifc:DirSort:SpecifiersSpread}
\end{figure}
%
with the following meanings for the fields
\begin{itemize}
    \item \field{locus} denotes the source location of this specifiers spread directive
    \item \field{specifiers} denotes the sequence of \grammar{decl-specifier}s in this specifiers spread directive
    \item \field{targets} denotes the sequence of \grammar{init-declarator}s in this specifiers spread directive
\end{itemize}


\partition{dir.specifiers-spread}


\subsection{\valueTag{DirSort::Tuple}}
\label{sec:ifc:DirSort:Tuple}

A \type{DirIndex} value with tag \valueTag{DirSort::Tuple} represents an abstract reference to a group of directives treated as a single directive.
The \field{index} field is an index into the directive tuple partition.
Each entry in that partition is a structure with the following layout
%
\begin{figure}[H]
    \centering
    \structure{
        \DeclareMember{locus}{SourceLocation} \\
        \DeclareMember{elements}{DirIndexSequence} \\
    }
    \caption{Structure of a Directive tuple}
    \label{fig:ifc:DirSort:Tuple}
\end{figure}
%
with the following meanings for the fields
\begin{itemize}
    \item \field{locus} denotes the source location of this tuple of directives
    \item \field{elements} denotes the description of the sequence of \type{DirIndex} values making up this tuple of directives 
\end{itemize}

\partition{dir.tuple}
