\label{sec:ifc-types}

Similar to declarations, types are also indicated by abstract type references.
They are values of type \type{TypeIndex}, with 32-bit precision and the
following layout
\begin{figure}[H]
  \centering
  \absref{5}{TypeSort}
  \caption{\type{TypeIndex}: Abstract reference of type}
  \label{fig:ifc-type-index}
\end{figure}

\begin{SortEnum}{TypeSort}
	\enumerator{VendorExtension}
	\enumerator{Fundamental}
	\enumerator{Designated}
	\enumerator{Tor}
	\enumerator{Syntactic}
	\enumerator{Expansion}
	\enumerator{Pointer}
	\enumerator{PointerToMember}
	\enumerator{LvalueReference}
	\enumerator{RvalueReference}
	\enumerator{Function}
	\enumerator{Method}
	\enumerator{Array}
	\enumerator{Typename}
	\enumerator{Qualified}
	\enumerator{Base}
	\enumerator{Decltype}
	\enumerator{Placeholder}
	\enumerator{Tuple}
	\enumerator{Forall}
	\enumerator{Unaligned}
	\enumerator{SyntaxTree}
\end{SortEnum}


\section{Type structures}
\label{sec:ifc-type-structures}

\subsection{\valueTag{TypeSort::VendorExtension}}
\label{sec:ifc:TypeSort:VendorExtension}

\partition{type.vendor-extension}

\subsection{\valueTag{TypeSort::Fundamental}}
\label{sec:ifc:TypeSort:Fundamental}

A \type{TypeIndex} abstract reference with tag \type{TypeSort::Fundamental} designates a fundamental type.
The \field{index} field of that abstract reference is an index into the fundamental type partition.
Each entry in that partition is a structure with the following components:
%
\begin{figure}[H]
	\centering
	\structure{
		\DeclareMember{basis}{TypeBasis} \\
		\DeclareMember{precision}{TypePrecision} \\
		\DeclareMember{sign}{TypeSign} \\
		\DeclareMember{<padding>}{uint8\_t} \\
	}
	\caption{Structure of a fundamental type}
	\label{fig:ifc-fundamental-type-structure}
\end{figure}
%
The \field{basis} field designates the fundamental basis of the fundamental type.
The \field{precision} field designates the "bit precision" variant of the basis type.
The \field{sign} field indicates the "sign" variant of the basis type.

\partition{type.fundamental}

\subsubsection{Fundamental type basis}
\label{sec:ifc-fundamental-type-basis}

The fundamental types are made out a small set of type basis defined as:
%
\begin{typedef}{TypeBasis}{}
	enum class TypeBasis : uint8_t {
		Void,
		Bool,
		Char,
		Wchar_t,
		Int,
		Float,
		Double,
		Nullptr,
		Ellipsis,
		SegmentType,
		Class,
		Struct,
		Union,
		Enum,
		Typename,
		Namespace,
		Interface,
		Function,
		Empty,
		VariableTemplate,
		Concept,
		Auto,
		DecltypeAuto,
		Overload
	};
\end{typedef}
%
with the following meaning:
\begin{itemize}
  \item \code{TypeBasis::Void}: fundamental type \code{void}
  \item \code{TypeBasis::Bool}: fundamental type \code{bool}
  \item \code{TypeBasis::Char}: fundamental type \code{char}
  \item \code{TypeBasis::Wchar\_t}: fundamental type \code{wchar\_t}
  \item \code{TypeBasis::Int}: fundamental type \code{int}
  \item \code{TypeBasis::Float}: fundamental type \code{float}
  \item \code{TypeBasis::Double}: fundamental type \code{double}
  \item \code{TypeBasis::Nullptr}: fundamental type \code{decltype(nullptr)}
  \item \code{TypeBasis::Ellipsis}: fundamental type denoted by \code{...}
  \item \code{TypeBasis::SegmentType}.  Note: this basis member is subject to removal in future revision.
  \item \code{TypeBasis::Class}: fundamental type \code{class}
  \item \code{TypeBasis::Struct}: fundamental type \code{struct}
  \item \code{TypeBasis::Union}: fundamental type \code{union}
  \item \code{TypeBasis::Enum}: fundamental type \code{enum}
  \item\label{sec:ifc:TypeBasis:Typename} \code{TypeBasis::Typename}: fundamental concept \code{typename}
  \item \code{TypeBasis::Namespace}: fundamental type \code{namespace}
  \item \code{TypeBasis::Interface}: fundamental type \code{\_\_interface}
  \item \code{TypeBasis::Function}: fundamental concept of function.  Note: this basis member is object to removal in future revision.
  \item \code{TypeBasis::Empty}: fundamental type resulting from an empty pack expansion
  \item \code{TypeBasis::VariableTemplate}: concept of variable template.  Note: this basis member is subject to removal in future revision.
  \item \code{TypeBasis::Concept}: fundamental type of a \code{concept} expression.
  \item \code{TypeBasis::Auto}: type placeholder \code{auto}
  \item \code{TypeBasis::DecltypeAuto} type placeholder \code{decltype(auto)}
  \item \code{TypeBasis::Overload}:  fundamental type for an overload set.
\end{itemize}

\note{The current set of type basis is subject to change.}

\subsubsection{Fundamental type precision}
\label{sec:ifc-fundamental-type-precision}

The bit precision of a fundamental type is a value of type \type{TypePrecision} defined as follows:
%
\begin{typedef}{TypePrecision}{}
	enum class TypePrecision : uint8_t {
		Default,
		Short,
		Long,
		Bit8,
		Bit16,
		Bit32,
		Bit64,
		Bit128,
	};
\end{typedef}
%
with the following meaning:
\begin{itemize}
  \item \code{TypePrecision::Default}: the default precision of the basis type
  \item \code{TypePrecision::Short}: the \code{short} variant of the basis type
  \item \code{TypePrecision::Long}: the \code{long} variant of the basis type
  \item \code{TypePrecision::Bit8}: the 8-bit variant of the basis type
  \item \code{TypePrecision::Bit16}: the 16-bit variant of the basis type
  \item \code{TypePrecision::Bit32}: the 32-bit variant of the basis type
  \item \code{TypePrecision::Bit64}: the 64-bit variant of the basis type
  \item \code{TypePrecision::Bit128}: the 128-bit variant of the basis type
\end{itemize}


\subsubsection{Fundamental type sign}
\label{sec:ifc-fundamental-type-sign}
The sign of a fundamental type is expressed as a value of type \type{TypeSign} defined as follows:
%
\begin{typedef}{TypeSign}{}
	enum class TypeSign : uint8_t {
		Plain,
		Signed,
		Unsigned,
	};
\end{typedef}
%
with the following meaning:
\begin{itemize}
  \item \code{TypeSign::Plain}: the plain sign of the basis type
  \item \code{TypeSign::Signed}: the signed variant of the basis type
  \item \code{TypeSign::Unsigned}: the unsigned variant of the basis type
\end{itemize}


\subsection{\valueTag{TypeSort::Designated}}
\label{sec:ifc:TypeSort:Designated}

A \type{TypeIndex} value with tag \valueTag{TypeSort::Designated} represents an
abstract reference to type denoted by a declaration name.  This is the typical use of a class name.
The \field{index} field is an index into the designated type partition.
Each entry in that partition is a structure with a single component: the \field{decl} field.
%
\begin{figure}[H]
	\centering
	\structure{
		\DeclareMember{decl}{DeclIndex} \\
	}
	\caption{Structure of a designated type}
	\label{fig:ifc-designated-type-structure}
\end{figure}
%
The \field{decl} field denotes the declaration of the type.

\partition{type.designated}

\subsection{\valueTag{TypeSort::Tor}}
\label{sec:ifc:TypeSort:Tor}

A \type{TypeIndex} value with \valueTag{TypeSort::Tor} represents an abstract reference to
the type of a constructor declaration.  ISO C++ does not define the type of a constructor -- it even denies
a constructor has a name.  However, for regularly and handling of the representation of 
template declaration, it is simpler to assign types to constructors.
The \field{index} field of that abstract reference is an index into the partition of tor types.
Each entry in that partition is a structure with the following layout
%
\begin{Structure}
	\structure{
		\DeclareMember{source}{TypeIndex} \\
		\DeclareMember{eh\_spec}{NoexceptSpecification} \\
		\DeclareMember{convention}{CallingConvention} \\
	}
	\caption{Structure of a tor type}
	\label{fig:ifc:TypeSort:Tor}
\end{Structure}
%
The meaning of the fields is as follows
\begin{itemize}
	\item \field{source} denotes the sequence of parameter types in the declaration of the corresponding constructor.
	\item \field{eh\_spec} denotes the exception specification associated with the constructor declaration this type associates with.
	\item \field{convention} denotes the calling convention associated with the constructor declaration.
\end{itemize}

\partition{type.tor}

\subsection{\valueTag{TypeSort::Syntactic}}
\label{sec:ifc:TypeSort:Syntactic}

A \type{TypeIndex} value with tag \valueTag{TypeSort::Syntactic} represents
an abstract reference to type expressed at the C++ source-level as a type-id.
Typical examples include a template-id designating a specialization.
The \field{index} field is an index into the syntactic type partition.
Each entry in that partition is a structure with a single component: the \field{expr} field.
%
\begin{figure}[H]
	\centering
	\structure{
		\DeclareMember{expr}{ExprIndex} \\
	}
	\caption{Structure of a syntactic type}
	\label{fig:ifc-syntactic-type-structure}
\end{figure}
%
The \field{expr} field denotes the C++ source-level type expression.

\partition{type.syntactic}

\subsection{\valueTag{TypeSort::Expansion}}
\label{sec:ifc:TypeSort:Expansion}

A \type{TypeIndex} abstract reference with tag \valueTag{TypeSort::Expansion} designates
an expansion of a pack type.  The \field{index} field is an index into the expansion
type partition.  Each entry in that partition is a structure with the following layout
%
\begin{figure}[H]
	\centering
	\structure{
		\DeclareMember{pack}{TypeIndex} \\
		\DeclareMember{mode}{ExpansionMode} \\
	}
	\caption{Structure of an expansion type}
	\label{fig:fic-expansion-type-structure}
\end{figure}
%
The \field{pack} field designates the type form under expansion.
The \field{mode} field denotes the mode of expansion.

\partition{type.expansion}

\subsubsection{Pack expansion mode}
\label{sec:ifc-pack-expansion-mode}

During pack expansion, a template parameter can be fully expanded, or only 
partially expanded.  The mode is indicated by a value of type
%
\begin{typedef}{ExpansionMode}{}
	enum class ExpansionMode : uint8_t {
		Full, 
		Partial
	};
\end{typedef}
%


\subsection{\valueTag{TypeSort::Pointer}}
\label{sec:ifc:TypeSort:Pointer}

A \type{TypeIndex} value with tag \valueTag{TypeSort::Pointer} represents
an abstract reference to a pointer type.
The \field{index} is an index into the pointer type partition.
Each entry in that partition is a structure with one component: the \field{pointee} field.
%
\begin{figure}[H]
	\centering
	\structure{
		\DeclareMember{pointee}{TypeIndex} \\
	}
	\caption{Structure of a pointer type}
	\label{fig:ifc-pointer-type-structure}
\end{figure}
%
The \field{pointee} field denotes the type pointed-to: any valid C++ type.

\partition{type.pointer}

\subsection{\valueTag{TypeSort::PointerToMember}}
\label{sec:ifc:TypeSort:PointerToMember}

A \type{TypeIndex} value with tag \valueTag{TypeSort::PointerToMember}
represents an abstract reference to a C++ source-level pointer-to-nonstatic-data member
type.
The \field{index} field is an index into the pointer-to-member type partition.
Each entry in that partition is a structure with two components: the \field{scope} field,
and the \field{member} field.
%
\begin{figure}[H]
	\centering
	\structure{
		\DeclareMember{scope}{TypeIndex} \\
		\DeclareMember{member}{TypeIndex} \\
	}
	\caption{Structure of a pointer-to-member type}
	\label{fig:ifc-pointer-to-member-type-structure}
\end{figure}
%
The \field{scope} field denotes the enclosing class type.
The \field{member} field denotes the type of the member.

\partition{type.pointer-to-member}


\subsection{\valueTag{TypeSort::LvalueReference}}
\label{sec:ifc:TypeSort:LvalueReference}

A \type{TypeIndex} value with tag \valueTag{TypeSort::LvalueReference} represents
an abstract reference to a C++ classic reference type, e.g. \code{int&}.
The \field{index} field is an index into the lvalue-reference type partition.
Each entry in that partition is a structure with one component: the \field{referee} field.
%
\begin{figure}[H]
	\centering
	\structure{
		\DeclareMember{referee}{TypeIndex} \\
	}
	\caption{Structure of an lvalue-reference type}
	\label{fig:ifc-lvalue-reference-type-structure}
\end{figure}
%
The \field{referee} field denotes the type referred-to: it is any valid C++ type,
including function type.

\partition{type.lvalue-reference}


\subsection{\valueTag{TypeSort::RvalueReference}}
\label{sec:ifc:TypeSort:RvalueReference}

A \type{TypeIndex} value with tag \valueTag{TypeSort::RvalueReference} represents
an abstract reference to a C++ rvalue-reference type, e.g. \code{int&&}.
The \field{index} field is an index into the rvalue-reference type partition.
Each entry in that partition is a structure with one component: the \field{referee} field.
%
\begin{figure}[H]
	\centering
	\structure{
		\DeclareMember{referee}{TypeIndex} \\
	}
	\caption{Structure of an rvalue-reference type}
	\label{fig:ifc-rvalue-reference-type-structure}
\end{figure}
%
The \field{referee} field denotes the type referred-to: it is any valid C++ type.

\partition{type.rvalue-reference}

\subsection{\valueTag{TypeSort::Function}}
\label{sec:ifc:TypeSort:Function}

A \type{TypeIndex} with tag \valueTag{TypeSort::Function} represents
an abstract reference to a C++ source-level function type.
The \field{index} field is an index into the function type partition.
Each entry in that partition is a structure with the following components:
a \field{target} field, a \field{source} field, an \field{eh\_spec} field,
a \field{convention} field, a \field{traits} field.
%
\begin{figure}[H]
	\centering
	\structure[text width=15em]{
		\DeclareMember{target}{TypeIndex} \\
		\DeclareMember{source}{TypeIndex} \\
		\DeclareMember{eh\_spec}{NoexceptSpecification} \\
		\DeclareMember{convention}{CallingConvention} \\
		\DeclareMember{traits}{FunctionTypeTraits} \\
	}
	\caption{Structure of a function type}
	\label{fig:ifc-function-type-structure}
\end{figure}
%
The \field{target} field denotes the return type of the function type.
The \field{source} field denotes the parameter type list.  A null \field{source} value indicates no parameter type. If the function type has at most
one parameter type, the \field{source} denotes that type. Otherwise, it is a tuple type made
of all the parameter types.
The \field{eh\_spec} denotes the C++ source-level noexcept-specification.
The \field{convention} field denotes the calling convention of the function type.
The \field{traits} field denotes additional function type traits.

\partition{type.function}

\subsubsection{Function type traits}
\label{sec:ifc-function-type-traits}

Certain function types properties are expressed as value of bitmask type \type{FunctionTypeTraits} defined as:
%
\begin{typedef}{FunctionTypeTraits}{}
	enum class FunctionTypeTraits : uint8_t {
		None			= 0,
		Const			= 1 << 0,
		Volatile			= 1 << 1,
		Lvalue			= 1 << 2,
		Rvalue			= 1 << 3,
	};
\end{typedef}
%
with the following meaning
\begin{itemize}
	\item \code{FunctionTypeTraits::None}: the function type is that of a function that is not non-static member (either a non-member function
		or a static member function).
	\item \code{FunctionTypeTraits::Const}: the function type is that of a function that is a non-static member function declared with a \code{const}
		qualifier.
	\item \code{FunctionTypeTraits::Volatile}: the function type is that of a function that is a non-static member function declared with a \code{volatile}
		qualifier.
	\item \code{FunctionTypeTraits::Lvalue}: the function type is that of a function that is a non-static member function declared with an lvalue (\code{\&})
		qualifier.
	\item \code{FunctionTypeTraits::Rvalue}: the function type is that of a function that is a non-static member function declared with an rvalue (\code{\&\&})
		qualifier.
\end{itemize}


\subsubsection{Calling convention}
\label{sec:ifc-calling-convention}

Function calling conventions are expressed as values of type \type{CallingConvention} defined as:
%
\begin{typedef}{CallingConvention}{}
	enum class CallingConvention : uint8_t {
		Cdecl,
		Fast,
		Std,
		This,
		Clr,
		Vector,
		Eabi,
	};
\end{typedef}
%
with the following meaning
\begin{itemize}
	\item \code{CallingConvention::Cdecl}: the function is declared with \code{\_\_cdecl}.
	\item \code{CallingConvention::Fast}: the function is declared with \code{\_\_fastcall}.
	\item \code{CallingConvention::Std}: the function is declared with \code{\_\_stdcall}.
	\item \code{CallingConvention::This}: the function is declared with \code{\_\_thiscall}.
	\item \code{CallingConvention::Clr}: the function is declared with \code{\_\_clrcall}.
	\item \code{CallingConvention::Vector}: the function is declared with \code{\_\_vectorcall}.
	\item \code{CallingConvention::Eabi}: the function is declared with \code{\_\_eabi}.
\end{itemize}


\note{The calling convention currently details only MSVC.}

\subsubsection{Noexcept specification}
\label{sec:ifc-noexcept-specification}

The noexcept-specification of a function is described by a structure with the following components:
%
\begin{figure}[H]
	\centering
	\structure{
		\DeclareMember{words}{SentenceIndex} \\
		\DeclareMember{sort}{NoexceptSort} \\
	}
	\caption{Structure of a noexcept-specification}
	\label{fig:ifc-noexception-specification-structure}
\end{figure}
%

The \field{words} field is meaningful only for function templates and complex \code{noexcept}-specification.
The \field{sort} describes the sort of noexcept-specification. 

\subsection{\valueTag{TypeSort::Method}}
\label{sec:ifc:TypeSort:Method}

A \type{TypeIndex} with tag \valueTag{TypeSort::Method} represents
an abstract reference to a C++ source-level non-static member function type.
The \field{index} field is an index into the method type partition.
Each entry in that partition is a structure with the following components:
a \field{target} field, a \field{source} field, a \field{scope} field, an \field{eh\_spec} field,
a \field{convention} field, a \field{traits} field.
%
\begin{figure}[H]
	\centering
	\structure[text width=15em]{
		\DeclareMember{target}{TypeIndex} \\
		\DeclareMember{source}{TypeIndex} \\
		\DeclareMember{scope}{TypeIndex} \\
		\DeclareMember{eh\_spec}{NoexceptSpecification} \\
		\DeclareMember{convention}{CallingConvention} \\
		\DeclareMember{traits}{FunctionTypeTraits} \\
	}
	\caption{Structure of a method type}
	\label{fig:ifc-method-type-structure}
\end{figure}
%
The \field{target} field denotes the return type of the function type.
The \field{source} field denotes the parameter type list.  The function type has at most
one parameter type, the \field{source} denotes that type. Otherwise, it is a tuple type made
of all the parameter types.
The \field{scope} denotes the C++ source-level enclosing class type.
The \field{eh\_spec} denotes the C++ source-level noexcept-specification.
The \field{convention} field denotes the calling convention of the function type.
The \field{traits} field denotes additional function type traits.


\partition{type.nonstatic-member-function}

\subsection{\valueTag{TypeSort::Array}}
\label{sec:ifc:TypeSort:Array}

A \type{TypeIndex} value with tag \valueTag{TypeSort::Array} represents
an abstract reference to a C++ source-level builtin array type.
The \field{index} field is an index into the array type partition.
Each entry in that partition is a structure with two components: a \field{element} field,
and a \field{extent} field.
%
\begin{figure}[H]
	\centering
	\structure{
		\DeclareMember{element}{TypeIndex} \\
		\DeclareMember{extent}{ExprIndex} \\
	}
	\caption{Structure of an array type}
	\label{fig:ifc-array-type-structure}
\end{figure}
%
The \field{element} denotes the element type of the array.
The \field{extent} denotes the number of elements in the array type along its outer most dimension.

\partition{type.array}


\subsection{\valueTag{TypeSort::Typename}}
\label{sec:ifc:TypeSort:Typename}

A \type{TypeIndex} value with tag \valueTag{TypeSort::Typename} represents
an abstract reference to a dependent type which at the C++ source-level is
written as type expression.
The \field{index} field is an index into the typename type partition.
Each entry in that partition is a structure with a single component: the \field{path} field.
%
\begin{figure}[H]
	\centering
	\structure{
		\DeclareMember{path}{ExprIndex} \\
	}
	\caption{Structure of a typename type}
	\label{fig:ifc-typename-type-structure}
\end{figure}
%
The \field{path} field denotes the expression designating the dependent type.

\partition{type.typename}

\subsection{\valueTag{TypeSort::Qualified}}
\label{sec:ifc:TypeSort:Qualified}

A \type{TypeIndex} value with tag \valueTag{TypeSort::Qualified} represents
an abstract reference to a C++ source-level cv-qualified type.
The \field{index} field is an index into the qualified type partition.
Each entry in that partition is a structure with two components: the \field{unqualified} field,
and the \field{qualifiers} field.
%
\begin{figure}[H]
	\centering
	\structure{
		\DeclareMember{unqualified}{TypeIndex} \\
		\DeclareMember{qualifiers}{Qualifiers} \\
	}
	\caption{Structure of a qualified type}
	\label{fig:ifc-qualified-type-structure}
\end{figure}
%
The \field{unqualified} field denotes the type without the toplevel cv-qualifiers.
The \field{qualifiers} field denotes the cv-qualifiers.

\partition{type.qualified}

\subsubsection{Type qualifiers}
\label{sec:ifc-type-qualifiers}

Standard type qualifiers are given values of type:
\begin{typedef}{Qualifiers}{}
	enum class Qualifier : uint8_t {
		None		= 0,
		Const		= 1 << 0,
		Volatile		= 1 << 1,
		Restrict		= 1 << 2,
	};
\end{typedef}
with the following meaning:
\begin{itemize}
	\item \code{Qualifiers::None}: No type qualifier
	\item \code{Qualifiers::Const}: the type is \code{const}-qualified
	\item \code{Qualifiers::Volatile}: the type is \code{volatile}-qualified.
	\item \code{Qualifiers::Restrict}: the type is \code{\_\_restrict}-qualified -- non-standard, but common extension.
\end{itemize}


\subsection{\valueTag{TypeSort::Base}}
\label{sec:ifc:TypeSort:Base}

A \type{TypeIndex} value with tag \valueTag{TypeSort::Base} represents an
abstract reference to a use of a type as a base-class (in a class inheritance)
at the C++ source-level.
The \field{index} field is an index into the base type partition.
Each entry in that partition is a structure with the following layout
%
\begin{figure}[H]
	\centering
	\structure{
		\DeclareMember{type}{TypeIndex} \\
		\DeclareMember{access}{Access} \\
		\DeclareMember{specifiers}{BaseTypeSpecifiers} \\
	}
	\caption{Structure of a base type}
	\label{fig:ifc-base-type-structure}
\end{figure}
%

The meanings of the fields are as follows:
\begin{itemize} 
	\item \field{type} denotes the type base being used as a base type.
	\item \field{access} denotes the access specifier in the inheritance path.
	\item \field{specifiers} indicates specifiers for the base type (\secref{sec:ifc-base-type-specifiers}).
\end{itemize}

\partition{type.base}

\note{This representation is subject to change}

\subsubsection{Base type specifiers}
\label{sec:ifc-base-type-specifiers}

Base type specifiers are bitmask values of type
\begin{typedef}{BaseTypeSpecifiers}{}
	enum class BaseTypeSpecifiers : uint8_t {
		None		= 0,
		Shared		= 0x01,
		Expanded    = 0x02,
	};
\end{typedef}
%
with the following meaning:
\begin{itemize}
	\item \code{BaseTypeSpecifiers::None}: no base type specifiers.
	\item \code{BaseTypeSpecifiers::Shared}: the base class is virtual.
	\item \code{BaseTypeSpecifiers::Expanded}: the base class is a pack expansion.
\end{itemize}

\note{Since type expansions are type designators, a base type that is a pack expansion should be designated by \valueTag{TypeSort::Expansion} (\secref{sec:ifc:TypeSort:Expansion}), not a bitmask value.}

\subsection{\valueTag{TypeSort::Decltype}}
\label{sec:ifc:TypeSort:Decltype}

A \type{TypeIndex} value with tag \valueTag{TypeSort::Decltype} represents
an abstract reference to the C++ source-level application of \code{decltype} to
an expression. Ideally, the operand should be represented by an \type{ExprIndex}
value.  However, in the current implementation the operand is represented
as an arbitrary sequence of tokens.
The \field{index} field is an index into the decltype type partition.
Each entry in that partition is a structure with a single component: the \field{words} field.
%
\begin{figure}[H]
	\centering
	\structure{
		\DeclareMember{expr}{SyntaxIndex} \\
	}
	\caption{Structure of an decltype type}
	\label{fig:ifc-decltype-type-structure}
\end{figure}
%
The \field{expr} field denotes the syntactic representation of the expression operand to \code{decltype}.

\partition{type.decltype}

\paragraph{Note:} The \code{decltype} representation will change in the future to change its operand representation to an expression tree.

\subsection{\valueTag{TypeSort::Placeholder}}
\label{sec:ifc:TypeSort:Placeholder}

A \type{TypeIndex} abstract reference with tag \valueTag{TypeSort::Placeholder}
 designates a placeholder type.
The \field{index} field is an index into the placeholder type partition.
Each entry in that partition is a structure with the following layout
%
\begin{figure}[H]
	\centering
	\structure{
		\DeclareMember{constraint}{ExprIndex} \\
		\DeclareMember{basis}{TypeBasis} \\
		\DeclareMember{elaboration}{TypeIndex} \\
	}
	\caption{Structure of a placeholder type}
	\label{fig:ifc-placeholder-type-structure}
\end{figure}
%
The \field{constraint} field designates the (generalized) predicate that the placeholder
shall satisfy -- at the input source level.
The \field{basis} field is a structure denoting the type basis
(\secref{sec:ifc-fundamental-type-basis}) pattern expected for placeholder type,
either \code{auto} or \code{decltype(auto)}.
The \field{elaboration} field, if non null, designates the deduced type corresponding to the placeholder type.

\partition{type.placeholder}

\subsection{\valueTag{TypeSort::Tuple}}
\label{sec:ifc:TypeSort:Tuple}

A \type{TypeIndex} value with tag \valueTag{TypeSort::Tuple} represents
an abstract reference to tuple type, i.e. finite sequence of types.
The \field{index} field is an index into the tuple type partition.
Each entry in that partition is a structure with two components: a \field{start} field,
and a \field{cardinality} field.
%
\begin{figure}[H]
	\centering
	\structure{
		\DeclareMember{start}{Index} \\
		\DeclareMember{cardinality}{Cardinality} \\
	}
	\caption{Structure of a tuple type}
	\label{fig:ifc-tuple-type-structure}
\end{figure}
%
The \field{start} field is an index into the type heap partition.  It designates the first element
in the tuple.
The \field{cardinality} field denotes the number of elements in the tuple.

\partition{type.tuple}

\subsection{\valueTag{TypeSort::Forall}}
\label{sec:ifc:TypeSort:Forall}

A \type{TypeIndex} value with tag \valueTag{TypeSort::Forall} designates a generalized $\forall$-type,
as can could be ascribed to a template declaration, even though a template declaration does not have a type in ISO C++. 
The \field{index} field is an index into the forall type partition.
Each entry in that partition is a structure with the following layout
%
\begin{figure}[H]
	\centering
	\structure{
		\DeclareMember{chart}{ChartIndex} \\
		\DeclareMember{subject}{TypeIndex} \\
	}
	\caption{Structure of a forall type}
	\label{fig:ifc-forall-type-structure}
\end{figure}
%
The \field{chart} field designates the set of parameter lists in the template declaration.
The \field{subject} field designated the type of the current instantiation.

\partition{type.forall}

\note{This generalized type is not yet used in current version of MSVC.  However, they will be used in future releases of MSVC.}


\subsection{\valueTag{TypeSort::Unaligned}}
\label{sec:ifc:TypeSort:Unaligned}

A \type{TypeIndex} value with tag \valueTag{TypeSort::Unaligned} represents
an abstract reference to a type with \code{__unaligned} specifier.
The \field{index} field is an index into the unaligned type partition.
Each entry in that partition is a structure with a single component: the \field{type} field.
%
\begin{figure}[H]
	\centering
	\structure{
		\DeclareMember{type}{TypeIndex} \\
	}
	\caption{Structure of an unaligned type}
	\label{fig:ifc-unaligned-type-structure}
\end{figure}
%

\partition{type.unaligned}

\paragraph{Note:} An \code{__unaligned} type is an MSVC extension, with no
clearly defined semantics.  This representation should be expressed in a more
regular framework of vendor-extension types.


\subsection{\valueTag{TypeSort::SyntaxTree}}
\label{sec:ifc:TypeSort:SyntaxTree}

A \type{TypeIndex} value with tag \valueTag{TypeSort::SyntaxTree} represents an abstract 
reference to a type designated by a raw syntax construct (\secref{sec:ifc-syntax-tree-table}).
The \field{index} field is an index into the syntax tree type partition.
Each entry in that partition is a structure with a single component: the \field{syntax} field:
%
\begin{figure}[H]
	\centering
	\structure{
		\DeclareMember{syntax}{SyntaxIndex} \\
	}
	\caption{Structure of a syntax tree type}
	\label{fig:ifc-syntax-tree-type-structure}
\end{figure}
%

\partition{type.syntax-tree}


