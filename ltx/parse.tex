\label{sec:ifc-syntax-tree-table}

The syntax tree part of the compiler is still a work in progress.  The MSVC compiler front-end is going through a long
internal overhaul with a mixture of representations, none satisfactory.  The latest being so called ``parse trees'' that tries
to capture the syntax in the input source code as written.  Clearly, that is bound to both complexity and instability.
The front-end is moving away from ``parse trees'', to a more abstract representation of syntax fragments, but none of that work is complete
in the MSVC releases yet.

Each syntax fragment in the ``parse trees'' can be referred by an abstract reference of type \type{SyntaxTree} defined
as follows
\begin{figure}[H]
	\centering
	\absref{7}{SyntaxSort}
	\caption{\code{SyntaxIndex}: Abstract reference of syntax fragment}
	\label{fig:ifc-syntax-index}
\end{figure}

\begin{SortEnum}{SyntaxSort}
	\enumerator{VendorExtension}
	\enumerator{SimpleTypeSpecifier}
	\enumerator{DecltypeSpecifier}
	\enumerator{PlaceholderTypeSpecifier}
	\enumerator{TypeSpecifierSeq}
	\enumerator{DeclSpecifierSeq}
	\enumerator{VirtualSpecifierSeq}
	\enumerator{NoexceptSpecification}
	\enumerator{ExplicitSpecifier}
	\enumerator{EnumSpecifier}
	\enumerator{EnumeratorDefinition}
	\enumerator{ClassSpecifier}
	\enumerator{MemberSpecification}
	\enumerator{MemberDeclaration}
	\enumerator{MemberDeclarator}
	\enumerator{AccessSpecifier}
	\enumerator{BaseSpecifierList}
	\enumerator{BaseSpecifier}
	\enumerator{TypeId}
	\enumerator{TrailingReturnType}
	\enumerator{Declarator}
	\enumerator{PointerDeclarator}
	\enumerator{ArrayDeclarator}
	\enumerator{FunctionDeclarator}
	\enumerator{ArrayOrFunctionDeclarator}
	\enumerator{ParameterDeclarator}
	\enumerator{InitDeclarator}
	\enumerator{NewDeclarator}
	\enumerator{SimpleDeclaration}
	\enumerator{ExceptionDeclaration}
	\enumerator{ConditionDeclaration}
	\enumerator{StaticAssertDeclaration}
	\enumerator{AliasDeclaration}
	\enumerator{ConceptDefinition}
	\enumerator{CompoundStatement}
	\enumerator{ReturnStatement}
	\enumerator{IfStatement}
	\enumerator{WhileStatement}
	\enumerator{DoWhileStatement}
	\enumerator{ForStatement}
	\enumerator{InitStatement}
	\enumerator{RangeBasedForStatement}
	\enumerator{ForRangeDeclaration}
	\enumerator{LabeledStatement}
	\enumerator{BreakStatement}
	\enumerator{ContinueStatement}
	\enumerator{SwitchStatement}
	\enumerator{GotoStatement}
	\enumerator{DeclarationStatement}
	\enumerator{ExpressionStatement}
	\enumerator{TryBlock}
	\enumerator{Handler}
	\enumerator{HandlerSeq}
	\enumerator{FunctionTryBlock}
	\enumerator{TypeIdListElement}
	\enumerator{DynamicExceptionSpec}
	\enumerator{StatementSeq}
	\enumerator{FunctionBody}
	\enumerator{Expression}
	\enumerator{FunctionDefinition}
	\enumerator{MemberFunctionDeclaration}
	\enumerator{TemplateDeclaration}
	\enumerator{RequiresClause}
	\enumerator{SimpleRequirement}
	\enumerator{TypeRequirement}
	\enumerator{CompoundRequirement}
	\enumerator{NestedRequirement}
	\enumerator{RequirementBody}
	\enumerator{TypeTemplateParameter}
	\enumerator{TemplateTemplateParameter}
	\enumerator{TypeTemplateArgument}
	\enumerator{NonTypeTemplateArgument}
	\enumerator{TemplateParameterList}
	\enumerator{TemplateArgumentList}
	\enumerator{TemplateId}
	\enumerator{MemInitializer}
	\enumerator{CtorInitializer}
	\enumerator{LambdaIntroducer}
	\enumerator{LambdaDeclarator}
	\enumerator{CaptureDefault}
	\enumerator{SimpleCapture}
	\enumerator{InitCapture}
	\enumerator{ThisCapture}
	\enumerator{AttributedStatement}
	\enumerator{AttributedDeclaration}
	\enumerator{AttributeSpecifierSeq}
	\enumerator{AttributeSpecifier}
	\enumerator{AttributeUsingPrefix}
	\enumerator{Attribute}
	\enumerator{AttributeArgumentClause}
	\enumerator{Alignas}
	\enumerator{UsingDeclaration}
	\enumerator{UsingDeclarator}
	\enumerator{UsingDirective}
	\enumerator{ArrayIndex}
	\enumerator{SEHTry}
	\enumerator{SEHExcept}
	\enumerator{SEHFinally}
	\enumerator{SEHLeave}
	\enumerator{TypeTraitIntrinsic}
	\enumerator{Tuple}
	\enumerator{AsmStatement}
	\enumerator{NamespaceAliasDefinition}
	\enumerator{Super}
	\enumerator{UnaryFoldExpression}
	\enumerator{BinaryFoldExpression}
	\enumerator{EmptyStatement}
	\enumerator{StructuredBindingDeclaration}
	\enumerator{StructuredBindingIdentifier}
	\enumerator{UsingEnumDeclaration}
\end{SortEnum}

\note{
	Syntax fragments are used only for the representation of the following syntactic constructs
	\begin{itemize}
		\item concept definition, or any constraint in dependent contexts
		\item template alias
		\item lambda in dependent contexts
		\item exception specification in template alias
		\item default argument for template parameters -- but not for default argument to functions or function templates.
	\end{itemize}
}

\section{Auxiliary types}
\label{sec:ifc:syntax-auxiliary-types}

\subsection{\type{KeywordSyntax}}
\label{sec:ifc:syntax-keyword-syntax}

%
\begin{figure}[H]
	\centering
	\structure{
		\DeclareMember{locus}{SourceLocation} \\
		\DeclareMember{value}{KeywordSort} \\
	}
	\caption{Structure of a keyword syntax}
	\label{fig:ifc:syntax-keyword-syntax}
\end{figure}
%

\subsubsection{\type{KeywordSort}}
\label{sec:ifc:KeywordSort}

\defineBitsize{KeywordSort}{8}

\begin{SortEnum}{KeywordSort}
	\enumerator{Nothing}
	\enumerator{Class}
	\enumerator{Struct}
	\enumerator{Union}
	\enumerator{Public}
    \enumerator{Protected}
	\enumerator{Private}
	\enumerator{Default}
	\enumerator{Delete}
    \enumerator{Mutable}
	\enumerator{Constexpr}
	\enumerator{Consteval}
	\enumerator{Typename}
\end{SortEnum}

\ifcdoc{Nothing}{KeywordSort} No source-level keyword.
\ifcdoc{Class}{KeywordSort} Source-level keyword ``\code{class}''.
\ifcdoc{Struct}{KeywordSort} Source-level keyword ``\code{struct}''.
\ifcdoc{Union}{KeywordSort} Source-level keyword ``\code{union}''.
\ifcdoc{Public}{KeywordSort} Source-level keyword ``\code{public}''.
\ifcdoc{Protected}{KeywordSort} Source-level keyword ``\code{protected}''.
\ifcdoc{Private}{KeywordSort} Source-level keyword ``\code{private}''.
\ifcdoc{Default}{KeywordSort} Source-level keywprd ``\code{default}''.
\ifcdoc{Delete}{KeywordSort} Source-level keyword ``\code{delete}''.
\ifcdoc{Mutable}{KeywordSort} Source-level keyword ``\code{mutable}''.
\ifcdoc{Constexpr}{KeywordSort} Source-level keywprd ``\code{constexpr}''.
\ifcdoc{Consteval}{KeywordSort} Source-level keyword ``\code{consteval}''.
\ifcdoc{Typename}{KeywordSort} Source-level keyword ``\code{typename}''.

\section{Syntax tree structures}
\label{sec:ifc:SyntaxSort-structures}

\subsection{\valueTag{SyntaxSort::VendorExtension}}
\label{sec:ifc:SyntaxSort:VendorExtension}

\partition{syntax.vendor-extension}

\subsection{\valueTag{SyntaxSort::SimpleTypeSpecifier}}
\label{sec:ifc:SyntaxSort:SimpleTypeSpecifier}

\begin{figure}[H]
	\centering
	\structure{
		\DeclareMember{type}{TypeIndex} \\
		\DeclareMember{expr}{ExprIndex} \\
		\DeclareMember{locus}{SourceLocation} \\
	}
	\caption{Structure of a simple-type-specifier syntax}
	\label{fig:ifc:SyntaxSort:SimpleTypeSpecifier}
\end{figure}

\partition{syntax.simple-type-specifier}

\subsection{\valueTag{SyntaxSort::DecltypeSpecifier}}
\label{sec:ifc:SyntaxSort:DecltypeSpecifier}

\begin{figure}[H]
	\centering
	\structure[text width = 15em]{
		\DeclareMember{expr}{ExprIndex} \\
		\DeclareMember{decltype\_keyword}{SourceLocation} \\
		\DeclareMember{left\_paren}{SourceLocation} \\
		\DeclareMember{right\_paren}{SourceLocation} \\
	}
	\caption{Structure of the decltype-specifier syntax-tree structure}
	\label{fig:ifc:SyntaxSort:DecltypeSpecifier}
\end{figure}

\partition{syntax.decltype-specifier}

\subsection{\valueTag{SyntaxSort::PlaceholderTypeSpecifier}}
\label{sec:ifc:SyntaxSort:PlaceholderTypeSpecifier}

A \type{SyntaxIndex} abstract reference with tag 
\valueTag{SyntaxSort::PlaceholderTypeSpecifier}
designates a placeholder type given by the grammatical element described below.
The \field{index} field is an index into the grammatical placeholder type 
partition.  Each entry in that partition is a structure with the following
layout:
%
\begin{figure}[H]
	\centering
	\structure{
		\DeclareMember{type}{PlaceholderType} \\
		\DeclareMember{keyword}{SourceLocation} \\
		\DeclareMember{locus}{SourceLocation} \\
	}
	\label{fig:ifc:SyntaxSort:PlaceholderTypeSpecifier}
\end{figure}
%
The \field{type} field is a structure denoting the semantic placeholder type
 (\type{PlaceholderType}) as described in \sortref{Placeholder}{TypeSort}.  The \field{keyword} field
designates the source location of the syntax of the placeholder type
(denoted by \code{auto} or \code{decltype(auto)}).  Finally the \field{locus} field
denotes the source location of the entire syntax for the placeholder type;
this source location may be different from that desinated by \field{keyword}
if there are additional constraints (as indicated by the use of a concept type).

\partition{syntax.placeholder-type-specifier}


\subsection{\valueTag{SyntaxSort::TypeSpecifierSeq}}
\label{sec:ifc:SyntaxSort:TypeSpecifierSeq}

\begin{figure}[H]
	\centering
	\structure{
		\DeclareMember{typename}{SyntaxIndex} \\
		\DeclareMember{type}{TypeIndex} \\
		\DeclareMember{locus}{SourceLocation} \\
		\DeclareMember{qualifiers}{Qualifier} \\
		\DeclareMember{unshashed}{bool} \\
	}
	\label{fig:ifc:SyntaxSort:TypeSpecifierSeq}
\end{figure}

\partition{syntax.type-specifier-seq}

\subsection{\valueTag{SyntaxSort::DeclSpecifierSeq}}
\label{sec:ifc:SyntaxSort:DeclSpecifierSeq}

\begin{figure}[H]
	\centering
	\structure{
		\DeclareMember{type}{TypeIndex} \\
		\DeclareMember{typename}{SyntaxIndex} \\
		\DeclareMember{locus}{SourceLocation} \\
		\DeclareMember{storage\_class}{StorageClass} \\
		\DeclareMember{declspec}{SentenceIndex} \\
		\DeclareMember{explicit}{SyntaxIndex} \\
		\DeclareMember{qualifiers}{Qualifier} \\
	}
	\label{fig:ifc:SyntaxSort:DeclSpecifierSeq}
\end{figure}

\partition{syntax.decl-specifier-seq}

\subsection{\valueTag{SyntaxSort::VirtualSpecifierSeq}}
\label{sec:ifc:SyntaxSort:VirtualSpecifierSeq}

\begin{figure}[H]
	\centering
	\structure{
		\DeclareMember{locus}{SourceLocation} \\
		\DeclareMember{final}{SourceLocation} \\
		\DeclareMember{override}{SourceLocation} \\
		\DeclareMember{pure}{bool} \\
	}
	\label{fig:ifc:SyntaxSort:VirtualSpecifierSeq}
\end{figure}

\partition{syntax.virtual-specifier-seq}


\subsection{\valueTag{SyntaxSort::NoexceptSpecification}}
\label{sec:ifc:SyntaxSort:NoexceptSpecification}

\begin{figure}[H]
	\centering
	\structure{
		\DeclareMember{expr}{SyntaxIndex} \\
		\DeclareMember{locus}{SourceLocation} \\
		\DeclareMember{left\_paren}{SourceLocation} \\
		\DeclareMember{right\_paren}{SourceLocation} \\
	}
	\label{fig:ifc:SyntaxSort:NoexceptSpecification}
\end{figure}

\partition{syntax.noexcept-specification}


\subsection{\valueTag{SyntaxSort::ExplicitSpecifier}}
\label{sec:ifc:SyntaxSort:ExplicitSpecifier}

\begin{figure}[H]
	\centering
	\structure{
		\DeclareMember{condition}{ExprIndex} \\
		\DeclareMember{locus}{SourceLocation} \\
		\DeclareMember{left\_paren}{SourceLocation} \\
		\DeclareMember{right\_paren}{SourceLocation} \\
	}
	\label{fig:ifc:SyntaxSort:ExplicitSpecifier}
\end{figure}

\partition{syntax.explicit-specifier}


\subsection{\valueTag{SyntaxSort::EnumSpecifier}}
\label{sec:ifc:SyntaxSort:EnumSpecifier}

\begin{figure}[H]
	\centering
	\structure{
		\DeclareMember{name}{ExprIndex} \\
		\DeclareMember{class\_key}{KeywordSyntax} \\
		\DeclareMember{enumerators}{SyntaxIndex} \\
		\DeclareMember{base}{SyntaxIndex} \\
		\DeclareMember{locus}{SourceLocation} \\
		\DeclareMember{colon}{SourceLocation} \\
		\DeclareMember{left\_brace}{SourceLocation} \\
		\DeclareMember{right\_brace}{SourceLocation} \\
	}
	\label{fig:ifc:SyntaxSort:EnumSpecifier}
\end{figure}

\partition{syntax.enum-specifier}

\subsection{\valueTag{SyntaxSort::EnumeratorDefinition}}
\label{sec:ifc:SyntaxSort:EnumeratorDefinition}

\begin{figure}[H]
	\centering
	\structure{
		\DeclareMember{name}{TextOffset} \\
		\DeclareMember{initializer}{ExprIndex} \\
		\DeclareMember{locus}{SourceLocation} \\
		\DeclareMember{equal}{SourceLocation} \\
		\DeclareMember{comma}{SourceLocation} \\
	}
	\label{fig:ifc:SyntaxSort:EnumeratorDefinition}
\end{figure}

\partition{syntax.enumerator-definition}


\subsection{\valueTag{SyntaxSort::ClassSpecifier}}
\label{sec:ifc:SyntaxSort:ClassSpecifier}

\begin{figure}[H]
	\centering
	\structure{
		\DeclareMember{name}{ExprIndex} \\
		\DeclareMember{class\_key}{KeywordSyntax} \\
		\DeclareMember{bases}{SyntaxIndex} \\
		\DeclareMember{members}{SyntaxIndex} \\
		\DeclareMember{left\_paren}{SourceLocation} \\
		\DeclareMember{right\_paren}{SourceLocation} \\
	}
	\label{fig:ifc:SyntaxSort:ClassSpecifier}
\end{figure}

\partition{syntax.class-specifier}


\subsection{\valueTag{SyntaxSort::MemberSpecification}}
\label{sec:ifc:SyntaxSort:MemberSpecification}

\begin{figure}[H]
	\centering
	\structure{
		\DeclareMember{member\_declarations}{SyntaxIndex} \\
	}
	\label{fig:ifc:SyntaxSort:MemberSpecification}
\end{figure}

\partition{syntax.member-specification}

\subsection{\valueTag{SyntaxSort::MemberDeclaration}}
\label{sec:ifc:SyntaxSort:MemberDeclaration}

\begin{figure}[H]
	\centering
	\structure{
		\DeclareMember{decl\_specifiers}{SyntaxIndex} \\
		\DeclareMember{declarators}{SyntaxIndex} \\
		\DeclareMember{semicolon}{SourceLocation} \\
	}
	\label{fig:ifc:SyntaxSort:MemberDeclaration}
\end{figure}

\partition{syntax.member-declaration}


\subsection{\valueTag{SyntaxSort::MemberDeclarator}}
\label{sec:ifc:SyntaxSort:MemberDeclarator}

\begin{figure}[H]
	\centering
	\structure{
		\DeclareMember{declarator}{SyntaxIndex} \\
		\DeclareMember{constraint}{SyntaxIndex} \\
		\DeclareMember{bitwidth}{ExprIndex} \\
		\DeclareMember{initializer}{ExprIndex} \\
		\DeclareMember{locus}{SourceLocation} \\
		\DeclareMember{colon}{SourceLocation} \\
		\DeclareMember{comma}{SourceLocation} \\
	}
	\label{fig:ifc:SyntaxSort:MemberDeclarator}
\end{figure}

\partition{syntax.member-declarator}

\subsection{\valueTag{SyntaxSort::AccessSpecifier}}
\label{sec:ifc:SyntaxSort:AccessSpecifier}

\begin{figure}[H]
	\centering
	\structure{
		\DeclareMember{access}{KeywordSyntax} \\
		\DeclareMember{colon}{SourceLocation} \\
	}
	\label{fig:ifc:SyntaxSort:AccessSpecifier}
\end{figure}

\partition{syntax.access-specifier}


\subsection{\valueTag{SyntaxSort::BaseSpecifierList}}
\label{sec:ifc:SyntaxSort:BaseSpecifierList}

\begin{figure}[H]
	\centering
	\structure{
		\DeclareMember{base\_specifiers}{SyntaxIndex} \\
		\DeclareMember{colon}{SourceLocation} \\
	}
	\label{fig:ifc:SyntaxSort:BaseSpecifierList}
\end{figure}

\partition{syntax.base-specifier-list}


\subsection{\valueTag{SyntaxSort::BaseSpecifier}}
\label{sec:ifc:SyntaxSort:BaseSpecifier}

\begin{figure}[H]
	\centering
	\structure{
		\DeclareMember{designator}{ExprIndex} \\
		\DeclareMember{access}{KeywordSyntax} \\
		\DeclareMember{virtual}{SourceLocation} \\
		\DeclareMember{locus}{SourceLocation} \\
		\DeclareMember{virtual}{SourceLocation} \\
		\DeclareMember{comma}{SourceLocation} \\
	}
	\label{fig:ifc:SyntaxSort:BaseSpecifier}
\end{figure}

\partition{syntax.base-specifier}


\subsection{\valueTag{SyntaxSort::TypeId}}
\label{sec:ifc:SyntaxSort:TypeId}

A \type{SyntaxIndex} value with tag \valueTag{SyntaxSort::TypeId} designates the source-level
syntactic contruct \grammar{type-id} which is the syntax
\grammar{type-specified-seq} optionally followed by \grammar{abstract-declarator}. 
The \field{index} field of that abstract reference is an index into the syntax of type-id partition.
Each entry of that partition is a structure with the following layout:
%
\begin{figure}[H]
	\centering
	\structure{
		\DeclareMember{type\_specifier}{SyntaxIndex} \\
		\DeclareMember{abstract\_declarator}{SyntaxIndex} \\
		\DeclareMember{locus}{SourceLocation} \\
	}
	\label{fig:ifc:SyntaxSort:TypeId}
\end{figure}
%
The \field{type\_specifier} field denotes the \grammar{type-specifier-seq} (\sortref{TypeSpecifierSeq}{SyntaxSort}).
The \field{abstract\_declarator} field, if not null, denotes the \grammar{abstract-declarator} part (\sortref{Declarator}{SyntaxSort}).
The \field{locus} designates the source location of the \grammar{type-id} construct.

\partition{syntax.type-id}


\subsection{\valueTag{SyntaxSort::TrailingReturnType}}
\label{sec:ifc:SyntaxSort:TrailingReturnType}

A \type{SyntaxIndex} value with tag \valueTag{SyntaxSort::TrailingReturnType} denotes the syntax tree of a source-level \grammar{trailing-return-type} construct.
The \field{index} field of that abstract reference is an index into the partition of trailing return type syntax tree.
Each entry of that partition is a structure with the following layout
%
\begin{figure}[H]
	\centering
	\structure{
		\DeclareMember{target}{SyntaxIndex} \\
		\DeclareMember{arrow}{SourceLocation} \\
	}
	\label{fig:ifc:SyntaxSort:TrailingReturnType}
\end{figure}
%
The \field{target} field denotes the syntax tree of the type specified in \grammar{trailing-return-type}.
The \field{arrow} field denotes the source location of the introducing arrow.


\partition{syntax.trailing-return-type}


\subsection{\valueTag{SyntaxSort::Declarator}}
\label{sec:ifc:SyntaxSort:Declarator}

A \type{SyntaxIndex} value with tag \valueTag{SyntaxSort::Declarator} designates a source level syntactic construct that
follows the grammar production \grammar{declarator} or \grammar{abstract-declarator}.
The \field{index} field of that abstract references is an index into the partition of declarator syntax trees.
Each entry of that partition is a structure with the following layout
%
\begin{figure}[H]
	\centering
	\structure[text width=15em]{
		\DeclareMember{pointer}{SyntaxIndex} \\
		\DeclareMember{parenthesized}{SyntaxIndex} \\
		\DeclareMember{array\_or\_function}{SyntaxIndex} \\
		\DeclareMember{trailing\_target}{SyntaxIndex} \\
		\DeclareMember{virtual\_specifiers}{SyntaxIndex} \\
		\DeclareMember{name}{ExprIndex} \\
		\DeclareMember{ellipsis}{SourceLocation} \\
		\DeclareMember{locus}{SourceLocation} \\
		\DeclareMember{qualifiers}{Qualifiers} \\
		\DeclareMember{convention}{CallingConvrntion} \\
		\DeclareMember{callable}{bool} \\
	}
	\label{fig:ifc:SyntaxSort:Declarator}
\end{figure}
%
In many ways, many fields of this structure form a discriminated union, meaning that not all of them are simultaneously meaningful and their interpretation are supposed to be mutually exclusive.
The \field{pointer} field, if non null, indicates that this is in fact a pointer-declarator (\sortref{PointerDeclarator}{SyntaxSort}).
The \field{parenthesized} field, if non null, indicates that outer parenthesis are used as part of the declarator.
The \field{array\_or\_function} field, if non null, denotes an array (\sortref{ArrayDeclarator}{SyntaxSort}) or function declarator (\sortref{FunctionDeclarator}{SyntaxSort}).
The \field{trailing\_target}, if non null, denotes a trailing return type (\sortref{TrailingReturnType}{SyntaxSort}) in the declarator.
The \field{virtual\_specifiers}, if non null, denotes the \grammar{virt-specifier-seq} grammatical element (\sortref{VirtualSpecifierSeq}{SyntaxSort}) in the source level construct.
The \field{name} field, if non null, denotes the name (\grammar{id-expression}) introduced by the declarator.
The \field{ellipsis} field, if non null, indicates that the expansion operatror \code{...} was used to introduce the \grammar{declarator-id}.
The \field{locus} field denotes the source location of the declarator.
The \field{qualifiers} field denotes the \grammar{cv-qualifier-seq} of the declarator.
The \field{convention} field denotes the calling convention specifier in the declarator (if this is a function declarator).
The \field{callable} field indicates that the declarator introduces a callable construct.


\partition{syntax.declarator}


\subsection{\valueTag{SyntaxSort::PointerDeclarator}}
\label{sec:ifc:SyntaxSort:PointerDeclarator}

\begin{figure}[H]
	\centering
	\structure{
		\DeclareMember{whole}{SyntaxIndex} \\
		\DeclareMember{next}{SyntaxIndex} \\
		\DeclareMember{locus}{SourceLocation} \\
		\DeclareMember{sort}{PointerDeclaratorSort} \\
		\DeclareMember{qualifiers}{Qualifier} \\
		\DeclareMember{convention}{CallingConvention} \\
		\DeclareMember{callable}{bool} \\
	}
	\label{fig:ifc:SyntaxSort:PointerDeclarator}
\end{figure}

\partition{syntax.pointer-declarator}

\subsection{\valueTag{SyntaxSort::ArrayDeclarator}}
\label{sec:ifc:SyntaxSort:ArrayDeclarator}

A \type{SyntaxIndex} value with tag \valueTag{SyntaxSort::ArrayDeclarator} denotes an array declarator syntax tree.
The \field{index} field of that abstract reference is an index into the array declarator syntax tree partition.
Each entry of that partition is a structure with the following layout
%
\begin{figure}[H]
	\centering
	\structure{
		\DeclareMember{bound}{ExprIndex} \\
		\DeclareMember{left\_bracket}{SourceLocation} \\
		\DeclareMember{right\_bracket}{SourceLocation} \\
	}
	\label{fig:ifc:SyntaxSort:ArrayDeclarator}
\end{figure}
%
The \field{bound} field, if non null, denotes the bound of the array.
The \field{left\_bracket} field denotes the source location of the open bracket.
The \field{right\_bracket} field denotes the source location of the closing bracket.

\partition{syntax.array-declarator}

\subsection{\valueTag{SyntaxSort::FunctionDeclarator}}
\label{sec:ifc:SyntaxSort:FunctionDeclarator}


A \type{SyntaxIndex} value with tag \valueTag{SyntaxSort::FunctionDeclarator} denotes a function declarator syntax tree.
The \field{index} field is an index into the function declarator syntax tree partition.
Each entry in that partition is a structure with the following layout
%
\begin{figure}[H]
	\centering
	\structure{
		\DeclareMember{parameters}{SyntaxIndex} \\
		\DeclareMember{eh\_spec}{SyntaxIndex} \\
		\DeclareMember{left\_paren}{SourceLocation} \\
		\DeclareMember{right\_paren}{SourceLocation} \\
	}
	\label{fig:ifc:SyntaxSort:FunctionDeclarator}
\end{figure}
%
The \field{parameters} denotes the parameter list of the function declarator.
The \field{eh\_spec} field, if non null, denotes the exception specification in the function declarator.
The \field{left\_paren} field denotes the location of the opening parenthesis of this function declarator.
The \field{right\_paren} field denotes the location of the closing parenthesis of this function declarator.

\partition{syntax.function-declarator}

\subsection{\valueTag{SyntaxSort::ArrayOrFunctionDeclarator}}
\label{sec:ifc:SyntaxSort:ArrayOrFunctionDeclarator}

A \type{SyntaxIndex} value with tag \valueTag{SyntaxSort::ArrayOrFunctionDeclarator} denotes the tower of array or function declarator syntax trees.
The \field{index} field is an index into the partition of array-or-function declarator syntax tree.
Each entry of that partition is a structure with the following layout
%
\begin{figure}[H]
	\centering
	\structure{
		\DeclareMember{declarator}{SyntaxIndex} \\
		\DeclareMember{next}{SyntaxIndex} \\
	}
	\label{fig:ifc:SyntaxSort:ArrayOrFunctionDeclarator}
\end{figure}
%
The \field{declarator} field denotes the outer declarator, while the \field{next} field
denotes the innner declarator.

\partition{syntax.array-or-function-declarator}

\subsection{\valueTag{SyntaxSort::ParameterDeclarator}}
\label{sec:ifc:SyntaxSort:ParameterDeclarator}

\begin{figure}[H]
	\centering
	\structure{
		\DeclareMember{decl\_specifiers}{SyntaxIndex} \\
		\DeclareMember{declarator}{SyntaxIndex} \\
		\DeclareMember{default}{ExprIndex} \\
		\DeclareMember{locus}{SourceLocation} \\
		\DeclareMember{sort}{ParameterSort} \\
	}
	\label{fig:ifc:SyntaxSort:ParameterDeclarator}
\end{figure}

\partition{syntax.parameter-declarator}


\subsection{\valueTag{SyntaxSort::InitDeclarator}}
\label{sec:ifc:SyntaxSort:InitDeclarator}

\begin{figure}[H]
	\centering
	\structure{
		\DeclareMember{declarator}{SyntaxIndex} \\
		\DeclareMember{constraint}{SyntaxIndex} \\
		\DeclareMember{initializer}{ExprIndex} \\
		\DeclareMember{comma}{SourceLocation} \\
	}
	\label{fig:ifc:SyntaxSort:InitDeclarator}
\end{figure}

\partition{syntax.init-declarator}

\subsection{\valueTag{SyntaxSort::NewDeclarator}}
\label{sec:ifc:SyntaxSort:NewDeclarator}

\begin{figure}[H]
	\centering
	\structure{
		\DeclareMember{declarator}{SyntaxIndex} \\
	}
	\label{fig:ifc:SyntaxSort:NewDeclarator}
\end{figure}

\partition{syntax.new-declarator}

\subsection{\valueTag{SyntaxSort::SimpleDeclaration}}
\label{sec:ifc:SyntaxSort:SimpleDeclaration}

\begin{figure}[H]
	\centering
	\structure{
		\DeclareMember{decl\_specifiers}{SyntaxIndex} \\
		\DeclareMember{declarators}{SyntaxIndex} \\
		\DeclareMember{locus}{SourceLocation} \\
		\DeclareMember{semicolon}{SourceLocartion} \\
	}
	\label{fig:ifc:SyntaxSort:SimpleDeclaration}
\end{figure}

\partition{syntax.simple-declaration}

\subsection{\valueTag{SyntaxSort::ExceptionDeclaration}}
\label{sec:ifc:SyntaxSort:ExceptionDeclaration}

\begin{figure}[H]
	\centering
	\structure{
		\DeclareMember{type\_specifiers}{SyntaxIndex} \\
		\DeclareMember{declarator}{SyntaxIndex} \\
		\DeclareMember{locus}{SourceLocation} \\
		\DeclareMember{ellipsis}{SourceLocation} \\
	}
	\label{fig:ifc:SyntaxSort:ExceptionDeclaration}
\end{figure}

\partition{syntax.exception-declaration}

\subsection{\valueTag{SyntaxSort::ConditionDeclaration}}
\label{sec:ifc:SyntaxSort:ConditionDeclaration}

\begin{figure}[H]
	\centering
	\structure{
		\DeclareMember{decl\_specifier}{SyntaxIndex} \\
		\DeclareMember{initializerion}{SyntaxIndex} \\
		\DeclareMember{locus}{SourceLocation} \\
	}
	\label{fig:ifc:SyntaxSort:ConditionDeclaration}
\end{figure}

\partition{syntax.condition-declaration}

\subsection{\valueTag{SyntaxSort::StaticAssertDeclaration}}
\label{sec:ifc:SyntaxSort:StaticAssertDeclaration}

\begin{figure}[H]
	\centering
	\structure{
		\DeclareMember{condition}{ExprIndex} \\
		\DeclareMember{message}{ExprIndex} \\
		\DeclareMember{locus}{SourceLocation} \\
		\DeclareMember{left\_paren}{SourceLocation} \\
		\DeclareMember{right\_paren}{SourceLocation} \\
		\DeclareMember{semicolon}{SourceLocation} \\
		\DeclareMember{comma}{SourceLocation} \\
	 }
	 \label{fig:ifc:SyntaxSort:StaticAssertDeclaration}
	\end{figure}

\partition{syntax.static-assert-declaration}


\subsection{\valueTag{SyntaxSort::AliasDeclaration}}
\label{sec:ifc:SyntaxSort:AliasDeclaration}

\begin{figure}[H]
	\centering
	\structure{
		\DeclareMember{name}{ExprIndex} \\
		\DeclareMember{aliasee}{SyntaxIndex} \\
		\DeclareMember{locus}{SourceLocation} \\
		\DeclareMember{equal}{SourceLocation} \\
		\DeclareMember{semicolon}{SourceLocation} \\
	}
	\label{fig:ifc:SyntaxSort:AliasDeclaration}
\end{figure}

\partition{syntax.alias-declaration}

\subsection{\valueTag{SyntaxSort::ConceptDefinition}}
\label{sec:ifc:SyntaxSort:ConceptDefinition}

\begin{figure}[H]
	\centering
	\structure{
		\DeclareMember{parameters}{SyntaxIndex} \\
		\DeclareMember{locus}{SourceLocation} \\
		\DeclareMember{name}{TextOffset} \\
		\DeclareMember{initializer}{ExprIndex} \\
		\DeclareMember{concept\_keyword}{SourceLocation} \\
		\DeclareMember{equal}{SourceLocation} \\
		\DeclareMember{semicolon}{SourceLocation} \\
	}
	\label{fig:ifc:SyntaxSort:ConceptDefinition}
\end{figure}

\partition{syntax.concept-definition}


\subsection{\valueTag{SyntaxSort::CompoundStatement}}
\label{sec:ifc:SyntaxSort:CompoundStatement}

\begin{Structure}
	\structure{
		\DeclareMember{pragma}{SentenceIndex} \\
		\DeclareMember{stmts}{SyntaxIndex} \\
		\DeclareMember{left\_curly}{SourceLocation} \\
		\DeclareMember{right\_curly}{SourceLocation} \\
	}
	\caption{Structure of a compound statement syntax tree}
	\label{fig:ifc:SyntaxSort:CompoundStatement}
\end{Structure}

\partition{syntax.compound-statement}


\subsection{\valueTag{SyntaxSort::ReturnStatement}}
\label{sec:ifc:SyntaxSort:ReturnStatement}

\begin{Structure}
	\structure{
		\DeclareMember{pragma}{SentenceIndex} \\
		\DeclareMember{expr}{ExprIndex} \\
		\DeclareMember{sort}{ReturnSort} \\
		\DeclareMember{return}{SourceLocation} \\
		\DeclareMember{semicolon}{SourceLocation} \\
	}
	\caption{Structure of a return statement syntax tree}
	\label{fig:ifc:SyntaxSort:ReturnStatement}
\end{Structure}

\partition{syntax.return-statement}

\paragraph{\type{ReturnSort}}

%
\begin{typedef}{ReturnSort}{}
	enum class ReturnSort : uint8_t {
		Return,
		Co_return,
	 };
\end{typedef}
%


\subsection{\valueTag{SyntaxSort::IfStatement}}
\label{sec:ifc:SyntaxSort:IfStatement}

\begin{Structure}
	\structure{
		\DeclareMember{pragma}{SentenceIndex} \\
		\DeclareMember{initialization}{SyntaxIndex} \\
		\DeclareMember{condition}{SyntaxIndex} \\
		\DeclareMember{condition}{ExprIndex} \\
		\DeclareMember{consequence}{SyntaxIndex} \\
		\DeclareMember{alternative}{SyntaxIndex} \\
		\DeclareMember{if}{SourceLocation} \\
		\DeclareMember{constexpr}{SourceLocation} \\
		\DeclareMember{else}{SourceLocation} \\
	}
	\caption{Structure of an \grammar{if-statement} syntax tree}
	\label{fig:ifc:SyntaxSort:IfStatement}
\end{Structure}

\partition{syntax.if-statement}


\subsection{\valueTag{SyntaxSort::WhileStatement}}
\label{sec:ifc:SyntaxSort:WhileStatement}

\begin{Structure}
	\structure{
		\DeclareMember{pragma}{SentenceIndex} \\
		\DeclareMember{condition}{ExprIndex} \\
		\DeclareMember{body}{SyntaxIndex} \\
		\DeclareMember{while}{SourceLocation} \\
	}
	\caption{Structure of an \grammar{while-statement} syntax tree}
	\label{fig:ifc:SyntaxSort:WhileStatement}
\end{Structure}

\partition{syntax.while-statement}


\subsection{\valueTag{SyntaxSort::DoWhileStatement}}
\label{sec:ifc:SyntaxSort:DoWhileStatement}

\begin{Structure}
	\structure{
		\DeclareMember{pragma}{SentenceIndex} \\
		\DeclareMember{condition}{ExprIndex} \\
		\DeclareMember{body}{SynyaxIndex} \\
		\DeclareMember{do}{SourceLocation} \\
		\DeclareMember{while}{SourceLocation} \\
		\DeclareMember{semicolon}{SourceLocation} \\
	}
	\caption{Structure of an \grammar{do-statement} syntax tree}
	\label{fig:ifc:SyntaxSort:DoWhileStatement}
\end{Structure}

\partition{syntax.do-statement}


\subsection{\valueTag{SyntaxSort::ForStatement}}
\label{sec:ifc:SyntaxSort:ForStatement}

\begin{Structure}
	\structure{
		\DeclareMember{pragma}{SentenceIndex} \\
		\DeclareMember{initialization}{SyntaxIndex} \\
		\DeclareMember{condution}{ExprIndex} \\
		\DeclareMember{continuation}{ExprIndex} \\
		\DeclareMember{body}{SyntaxIndex} \\
		\DeclareMember{for}{SourceLocation} \\
		\DeclareMember{left\_paren}{SourceLocation} \\
		\DeclareMember{right\_paren}{SourceLocation} \\
		\DeclareMember{semicolon}{SourceLocation} \\
	}
	\caption{Structure of an \grammar{for-statement} syntax tree}
	\label{fig:ifc:SyntaxSort:ForStatement}
\end{Structure}

\partition{syntax.for-statement}


\subsection{\valueTag{SyntaxSort::InitStatement}}
\label{sec:ifc:SyntaxSort:InitStatement}

\begin{Structure}
	\structure{
		\DeclareMember{pragma}{SentenceIndex} \\
		\DeclareMember{init}{SyntaxIndex} \\
	}
	\caption{Structure of an \grammar{init-statement} syntax tree}
	\label{fig:ifc:SyntaxSort:InitStatement}
\end{Structure}

\partition{syntax.init-statement}


\subsection{\valueTag{SyntaxSort::RangeBasedForStatement}}
\label{sec:ifc:SyntaxSort:RangeBasedForStatement}

\begin{Structure}
	\structure{
		\DeclareMember{pragma}{SentenceIndex} \\
		\DeclareMember{init}{SyntaxIndex} \\
		\DeclareMember{decl}{SyntaxIndex} \\
		\DeclareMember{initializer}{SyntaxIndex} \\
		\DeclareMember{body}{SynyaxIndex} \\
		\DeclareMember{for}{SourceLocation} \\
		\DeclareMember{left\_paren}{SourceLocation} \\
		\DeclareMember{right\_paren}{SourceLocation} \\
		\DeclareMember{colon}{SourceLocation} \\
	}
	\label{fig:ifc:SyntaxSort:RangeBasedForStatement}
\end{Structure}

\partition{syntax.range-based-for-statement}


\subsection{\valueTag{SyntaxSort::ForRangeDeclaration}}
\label{sec:ifc:SyntaxSort:ForRangeDeclaration}

\begin{Structure}
	\structure{
		\DeclareMember{specifiers}{SyntaxIndex} \\
		\DeclareMember{declarator}{SyntaxIndex} \\
	}
	\label{fig:ifc:SyntaxSort:ForRangeDeclaration}
\end{Structure}

\partition{syntax.for-range-declaration}


\subsection{\valueTag{SyntaxSort::LabeledStatement}}
\label{sec:ifc:SyntaxSort:LabeledStatement}

\begin{Structure}
	\structure{
		\DeclareMember{pragma}{SentencedIndex} \\
		\DeclareMember{label}{ExprIndex} \\
		\DeclareMember{stmt}{SyntaxIndex} \\
		\DeclareMember{locus}{SourceLocation} \\
		\DeclareMember{sort}{LabelSort} \\
	}
\end{Structure}

\partition{syntax.labeled-statement}

\paragraph{\type{LabelSort}}

\begin{typedef}{LabelSort}{}
	enum class LabelSort : uint8_t {
		Unknown,
		Case,
		Default,
		Label
	 };
\end{typedef}
%


\subsection{\valueTag{SyntaxSort::BreakStatement}}
\label{sec:ifc:SyntaxSort:BreakStatement}

\begin{Structure}
	\structure{
		\DeclareMember{break}{SourceLocation} \\
		\DeclareMember{semicolon}{SourceLocation} \\
	}
	\label{fig:ifc:SyntaxSort:BreakStatement}
\end{Structure}

\partition{syntax.break-statement}


\subsection{\valueTag{SyntaxSort::ContinueStatement}}
\label{sec:ifc:SyntaxSort:ContinueStatement}

\begin{Structure}
	\structure{
		\DeclareMember{continue}{SourceLocation} \\
		\DeclareMember{semicolon}{SourceLocation} \\
	}
	\label{fig:ifc:SyntaxSort:ContinueStatement}
\end{Structure}

\partition{syntax.continue-statement}


\subsection{\valueTag{SyntaxSort::SwitchStatement}}
\label{sec:ifc:SyntaxSort:SwitchStatement}

\begin{Structure}
	\structure{
		\DeclareMember{pragma}{SentenceIndex} \\
		\DeclareMember{init}{SyntaxIndex} \\
		\DeclareMember{condition}{SyntaxIndex} \\
		\DeclareMember{body}{SyntaxIndex} \\
		\DeclareMember{switch}{SourceLocation} \\
	}
	\label{fig:ifc:SyntaxSort:SwitchStatement}
\end{Structure}

\partition{syntax.switch-statement}

\subsection{\valueTag{SyntaxSort::GotoStatement}}
\label{sec:ifc:SyntaxSort:GotoStatement}

\begin{Structure}
	\structure{
		\DeclareMember{pragma}{SentenceIndex} \\
		\DeclareMember{target}{TextOffset} \\
		\DeclareMember{locus}{SourceLocation} \\
		\DeclareMember{label}{SourceLocation} \\
		\DeclareMember{semicolon}{SourceLocation} \\
	}
	\label{fig:ifc:SyntaxSort:GotoStatement}
\end{Structure}

\partition{syntax.goto-statement}

\subsection{\valueTag{SyntaxSort::DeclarationStatement}}
\label{sec:ifc:SyntaxSort:DeclarationStatement}

\begin{Structure}
	\structure{
		\DeclareMember{pragma}{SentenceIndex} \\
		\DeclareMember{decl}{SyntaxIndex} \\
	}
	\label{fig:ifc:SyntaxSort:DeclarationStatement}
\end{Structure}

\partition{syntax.declaration-statement}

\subsection{\valueTag{SyntaxSort::ExpressionStatement}}
\label{sec:ifc:SyntaxSort:ExpressionStatement}

\begin{Structure}
	\structure{
		\DeclareMember{pragma}{SentenceIndex} \\
		\DeclareMember{expr}{ExprIndex} \\
		\DeclareMember{semicolon}{SourceLocation} \\
	}
	\label{fig:ifc:SyntaxSort:ExpressionStatement}
\end{Structure}

\partition{syntax.expression-statement}


\subsection{\valueTag{SyntaxSort::TryBlock}}
\label{sec:ifc:SyntaxSort:TryBlock}

\begin{Structure}
	\structure{
		\DeclareMember{pragma}{SentenceIndex} \\
		\DeclareMember{body}{SyntaxIndex} \\
		\DeclareMember{handlers}{SyntaxIndex} \\
		\DeclareMember{try}{SourceLocation} \\
	}
	\label{fig:ifc:SyntaxSort:TryBlock}
\end{Structure}

\partition{syntax.try-block}


\subsection{\valueTag{SyntaxSort::Handler}}
\label{sec:ifc:SyntaxSort:Handler}

\begin{Structure}
	\structure{
		\DeclareMember{pragma}{SentenceIndex} \\
		\DeclareMember{exception}{SyntaxIndex} \\
		\DeclareMember{body}{SyntaxIndex} \\
		\DeclareMember{catch}{SourceLocaion} \\
		\DeclareMember{left\_paren}{SourceLocation} \\
		\DeclareMember{right\_paren}{SourceLocation} \\
	}
	\label{fig:ifc:SyntaxSort:Handler}
\end{Structure}

\partition{syntax.handler}


\subsection{\valueTag{SyntaxSort::HandlerSeq}}
\label{sec:ifc:SyntaxSort:HandlerSeq}

\begin{Structure}
	\structure{
		\DeclareMember{handlers}{SyntaxIndex} \\
	}
	\label{fig:ifc:SyntaxSort:HandlerSeq}
\end{Structure}

\partition{syntax.handler-seq}


\subsection{\valueTag{SyntaxSort::FunctionTryBlock}}
\label{sec:ifc:SyntaxSort:FunctionTryBlock}

\begin{Structure}
	\structure{
		\DeclareMember{body}{SyntaxIndex} \\
		\DeclareMember{handlers}{SyntaxIndex} \\
		\DeclareMember{initializers}{SyntaxIndex} \\
	}
	\label{fig:ifc:SyntaxSort:FunctionTryBlock}
\end{Structure}

\partition{syntax.function-try-block}


\subsection{\valueTag{SyntaxSort::TypeIdListElement}}
\label{sec:ifc:SyntaxSort:TypeIdListElement}

\begin{Structure}
	\structure{
		\DeclareMember{type\_id}{SyntaxIndex} \\
		\DeclareMember{ellipsis}{SourceLocation} \\
	}
	\label{fig:ifc:SyntaxSort:TypeIdListElement}
\end{Structure}

\partition{syntax.type-id-list-element}


\subsection{\valueTag{SyntaxSort::DynamicExceptionSpec}}
\label{sec:ifc:SyntaxSort:DynamicExceptionSpec}

\begin{Structure}
	\structure{
		\DeclareMember{type\_list}{SyntaxIndex} \\
		\DeclareMember{throw}{SourceLocation} \\
		\DeclareMember{left\_paren}{SourceLocation} \\
		\DeclareMember{ellipsis}{SourceLocation} \\
		\DeclareMember{right\_paren}{SourceLocation} \\
	}
	\label{fig:ifc:SyntaxSort:DynamicExceptionSpec}
\end{Structure}

\partition{syntax.dynamic-exception-spec}


\subsection{\valueTag{SyntaxSort::StatementSeq}}
\label{sec:ifc:SyntaxSort:StatementSeq}

\begin{Structure}
	\structure{
		\DeclareMember{stmts}{SyntaxIndex} \\
	}
	\label{fig:ifc:SyntaxSort:StatementSeq}
\end{Structure}

\partition{syntax.statement-seq}


\subsection{\valueTag{SyntaxSort::FunctionBody}}
\label{sec:ifc:SyntaxSort:FunctionBody}

\begin{Structure}
	\structure{
		\DeclareMember{stmts}{SyntaxIndex} \\
		\DeclareMember{try\_block}{SyntaxIndex} \\
		\DeclareMember{initializers}{SyntaxIndex} \\
		\DeclareMember{generate}{KeywordSyntax} \\
		\DeclareMember{assign}{SourceLocation} \\
		\DeclareMember{semicolon}{SourceLocation} \\
	}
	\label{fig:ifc:SyntaxSort:FunctionBody}
\end{Structure}

\partition{syntax.function-body}


\subsection{\valueTag{SyntaxSort::Expression}}
\label{sec:ifc:SyntaxSort:Expression}

A \type{SyntaxIndex} value with tag \valueTag{SyntaxSort::Expression} designates an expression
(\secref{sec:ifc-exprs}), typically a non-dependent expression, within a parse tree.  The index of that
value designates an entry in the expression syntax partition.  Each entry of that partition has the following
layout:
%
\begin{figure}[H]
	\centering
	\structure{
		\DeclareMember{expression}{ExprIndex} \\
	}
	\caption{Structure of an expression as a parse tree}
	\label{fig:ifc:SyntaxSort:Expression}
\end{figure}
%
The \field{expression} field designates the representation of this expression.

\partition{syntax.expression}


\subsection{\valueTag{SyntaxSort::FunctionDefinition}}
\label{sec:ifc:SyntaxSort:FunctionDefinition}

\begin{Structure}
	\structure{
		\DeclareMember{stmts}{SyntaxIndex} \\
		\DeclareMember{try\_block}{SyntaxIndex} \\
		\DeclareMember{initializers}{SyntaxIndex} \\
		\DeclareMember{synthesis}{KeywordSyntax} \\
		\DeclareMember{assign}{SourceLocation} \\
		\DeclareMember{semicolon}{SourceLocation} \\
	}
	\label{fig:ifc:SyntaxSort:FunctionDefinition}
\end{Structure}

\partition{syntax.function-definition}


\subsection{\valueTag{SyntaxSort::MemberFunctionDeclaration}}
\label{sec:ifc:SyntaxSort:MemberFunctionDeclaration}

\begin{Structure}
	\structure{
		\DeclareMember{definition}{SyntaxIndex} \\
	}
	\label{fig:ifc:SyntaxSort:MemberFunctionDeclaration}
\end{Structure}

\partition{syntax.member-function-declaration}


\subsection{\valueTag{SyntaxSort::TemplateDeclaration}}
\label{sec:ifc:SyntaxSort:TemplateDeclaration}

\begin{Structure}
	\structure{
		\DeclareMember{parameters}{SyntaxIndex} \\
		\DeclareMember{subject}{SyntaxIndex} \\
		\DeclareMember{locus}{SourceLocation} \\
	}
	\caption{Structure of a template declaration syntax tree}
	\label{fig:ifc:SyntaxSort:TemplateDeclaration}
\end{Structure}
%
The meaning of the fields is as follows
\begin{itemize}
	\item \field{parameters} denotes the \grammar{template-parameter} list of this template declaration.
	\item \field{subject} denotes the declaration being parameterized by the template parameters.
	\item \field{locus} denotes the source location of this declaration.
\end{itemize}

\partition{syntax.template-declaration}


\subsection{\valueTag{SyntaxSort::RequiresClause}}
\label{sec:ifc:SyntaxSort:RequiresClause}

\begin{Structure}
	\structure{
		\DeclareMember{condition}{ExprIndex} \\
		\DeclareMember{locus}{SourceLocation} \\
	}
	\label{fig:ifc:SyntaxSort:RequiresClause}
\end{Structure}

\partition{syntax.requires-clause}


\subsection{\valueTag{SyntaxSort::SimpleRequirement}}
\label{sec:ifc:SyntaxSort:SimpleRequirement}

\begin{Structure}
	\structure{
		\DeclareMember{condition}{ExprIndex} \\
		\DeclareMember{locus}{SourceLocaion} \\
	}
	\label{fig:ifc:SyntaxSort:SimpleRequirement}
\end{Structure}

\partition{syntax.simple-requirement}


\subsection{\valueTag{SyntaxSort::TypeRequirement}}
\label{sec:ifc:SyntaxSort:TypeRequirement}

\begin{Structure}
	\structure{
		\DeclareMember{type}{ExprIndex} \\
		\DeclareMember{locus}{SourceLocation} \\
	}
	\label{fig:ifc:SyntaxSort:TypeRequirement}
\end{Structure}

\partition{syntax.type-requirement}


\subsection{\valueTag{SyntaxSort::CompoundRequirement}}
\label{sec:ifc:SyntaxSort:CompoundRequirement}

\begin{Structure}
	\structure{
		\DeclareMember{condition}{ExprIndex} \\
		\DeclareMember{constraint}{ExprIndex} \\
		\DeclareMember{locus}{SourceLocation} \\
		\DeclareMember{right\_curly}{SourceLocation} \\
		\DeclareMember{noexcept}{SourceLocaion} \\
	}
	\label{fig:ifc:SyntaxSort:CompoundRequirement}
\end{Structure}

\partition{syntax.compound-requirement}


\subsection{\valueTag{SyntaxSort::NestedRequirement}}
\label{sec:ifc:SyntaxSort:NestedRequirement}

\begin{Structure}
	\structure{
		\DeclareMember{condition}{ExprIndex} \\
		\DeclareMember{locus}{SourceLocation} \\
	}
	\label{fig:ifc:SyntaxSort:NestedRequirement}
\end{Structure}

\partition{syntax.nested-requirement}


\subsection{\valueTag{SyntaxSort::RequirementBody}}
\label{sec:ifc:SyntaxSort:RequirementBody}

\begin{Structure}
	\structure{
		\DeclareMember{requirements}{SyntaxIndex} \\
		\DeclareMember{locus}{SourceLocation} \\
		\DeclareMember{right\_curly}{SourceLocation} \\
	}
	\label{fig:ifc:SyntaxSort:RequirementBody}
\end{Structure}

\partition{syntax.requirement-body}


\subsection{\valueTag{SyntaxSort::TypeTemplateParameter}}
\label{sec:ifc:SyntaxSort:TypeTemplateParameter}

\begin{Structure}
	\structure{
		\DeclareMember{name}{TextOffset} \\
		\DeclareMember{constraint}{SyntaxIndex} \\
		\DeclareMember{argument}{SyntaxIndex} \\
		\DeclareMember{locus}{SourceLocation} \\
		\DeclareMember{ellipsis}{SourceLocation} \\
	}
	\caption{Structure of a type \grammar{template-parameter} syntax tree}
	\label{fig:ifc:SyntaxSort:TypeTemplateParameter}
\end{Structure}
%
The meaning of the fields is as follows:
\begin{itemize}
	\item \field{name} denotes the name introduced by this declaration of type \grammar{template-parameter}.
	\item \field{constraint} denotes any optional type constraint on this parameter.
	\item \field{argument} denotes the corresponding default argument (if any).
	\item \field{locus} denotes the source location of this declaration.
	\item \field{ellipsis}, if non null, denotes the source location of \code{...} indicating a parameter pack.
\end{itemize}

\partition{syntax.type-template-parameter}


\subsection{\valueTag{SyntaxSort::TemplateTemplateParameter}}
\label{sec:ifc:SyntaxSort:TemplateTemplateParameter}

\begin{Structure}
	\structure{
		\DeclareMember{name}{TextOffset} \\
		\DeclareMember{argument}{SyntaxIndex} \\
		\DeclareMember{parameters}{SyntaxIndex} \\
		\DeclareMember{locus}{SourceLocation} \\
		\DeclareMember{ellipsis}{SourceLocation} \\
		\DeclareMember{comma}{SourceLocation} \\
		\DeclareMember{key}{KeywordSyntax} \\
	}
	\caption{Structure of a template \grammar{template-parameter} syntax tree}
	\label{fig:ifc:SyntaxSort:TemplateTemplateParameter}
\end{Structure}
%
The meaning of the fields is as follows
\begin{itemize}
	\item \field{name} denotes the name introduced by this template \grammar{template-parameter} declaration.
	\item \field{argument}, if non null, denotes the corresponding default argument (if any).
	\item \field{parameters} denotes the parameter list associated with this template \grammar{template-parameter} declaration.
	\item \field{locus} denotes the source location of this declaration.
	\item \field{ellipsis}, if non null, denotes the source location of \code{...} indicating a parameter pack.
	\item \field{comma}, if non null, denotes the source location of the separating comma from the next \grammar{template-parameter} declaration.
	\item \field{key} denotes the keyword (either \code{typename} or \code{class}) used to declare this template \grammar{template-parameter}.
\end{itemize}

\partition{syntax.template-template-parameter}

\subsection{\valueTag{SyntaxSort::TypeTemplateArgument}}
\label{sec:ifc:SyntaxSort:TypeTemplateArgument}

%
\begin{Structure}
	\structure{
		\DeclareMember{argument}{ExprIndex} \\
		\DeclareMember{ellipsis}{SourceLocation} \\
		\DeclareMember{comma}{SourceLocation} \\
	}
	\caption{Structure of a type \grammar{template-argument}}
	\label{fig:ifc:SyntaxSort:TypeTemplateArgument}
\end{Structure}
%
The meaning of the fields is as follows
\begin{itemize}
	\item \field{argument} denotes the type \grammar{template-argument}.
	\item \field{ellipsis}, if non null, denotes the source location of \code{...} if this template argument is actually an expansion.
	\item \field{comma}, if non null, denotes the source location of the comma separating from the next template argument.
\end{itemize}

\partition{syntax.type-template-argument}

\subsection{\valueTag{SyntaxSort::NonTypeTemplateArgument}}
\label{sec:ifc:SyntaxSort:NonTypeTemplateArgument}

\begin{Structure}
	\structure{
		\DeclareMember{argument}{SyntaxIndex} \\
		\DeclareMember{ellipsis}{SourceLocation} \\
		\DeclareMember{comma}{SourceLocation} \\
	}
	\caption{Structure of a non-type \grammar{template-argument} syntax tree}
	\label{fig:ifc:SyntaxSort:NonTypeTemplateArgument}
\end{Structure}
%
The meaning of the fields is as follows
\begin{itemize}
	\item \field{argument} is an expression that denotes the non-type \grammar{template-argument}.
	\item \field{ellipsis}, if non null, denotes the source location of \code{...} if this template argument is actually an expansion.
	\item \field{comma}, if non null, denotes the source location of the comma separating from the next template argument.
\end{itemize}

\partition{syntax.non-type-template-argument}


\subsection{\valueTag{SyntaxSort::TemplateParameterList}}
\label{sec:ifc:SyntaxSort:TemplateParameterList}

\begin{Structure}
	\structure{
		\DeclareMember{parameters}{SyntaxIndex} \\
		\DeclareMember{clause}{SyntaxIndex} \\
		\DeclareMember{left\_angle}{SourceLocation} \\
		\DeclareMember{right\_angle}{SourceLocation} \\
	}
	\caption{Structure of a template parameter list syntax tree}
	\label{fig:ifc:SyntaxSort:TemplateParameterList}
\end{Structure}
%
The meaning of the fields is as follows:
\begin{itemize}
	\item \field{parameters} denotes the comma-separated sequence of \grammar{template-parameter} declarations.
	\item \field{left\_angle} denotes the source location of the opening \code{<} delimiter.
	\item \field{right\_angle} denotes the source location of the closing \code{>} delimiter.
\end{itemize}

\partition{syntax.template-parameter-list}


\subsection{\valueTag{SyntaxSort::TemplateArgumentList}}
\label{sec:ifc:SyntaxSort:TemplateArgumentList}

%
\begin{Structure}
	\structure{
		\DeclareMember{arguments}{SyntaxIndex} \\
		\DeclareMember{left\_angle}{SourceLocation} \\
		\DeclareMember{right\_angle}{SourceLocation} \\
	}
	\caption{Structure of a \grammar{template-argument-list} syntax tree}
	\label{fig:ifc:SyntaxSort:TemplateArgumentList}
\end{Structure}
%
The meaning of the fields is as follows:
\begin{itemize}
	\item \field{arguments} denotes the comma-separated sequence of template arguments.
	\item \field{left\_angle} denotes the source location of the opening \code{<} delimiter.
	\item \field{right\_angle} denotes the source location of the closing \code{>} delimiter.
\end{itemize}

\partition{syntax.template-argument-list}


\subsection{\valueTag{SyntaxSort::TemplateId}}
\label{sec:ifc:SyntaxSort:TemplateId}

\begin{Structure}
	\structure{
		\DeclareMember{name}{SyntaxIndex} \\
		\DeclareMember{symbol}{ExprIndex} \\
		\DeclareMember{arguments}{SyntaxIndex} \\
		\DeclareMember{locus}{SourceLocation} \\
		\DeclareMember{template}{SourceLocation} \\
	}
	\caption{Structure of a \grammar{template-id} syntax tree}
	\label{fig:ifc:SyntaxSort:TemplateId}
\end{Structure}
The meaning of the fields is as follows
\begin{itemize}
	\item \field{name}, when not null, designates the \grammar{template-name} used in the \grammar{template-id} term.  This is the case when the template is not known through a declaration.
	\item \field{symbol}, when not null, is an expression that denotes the declaration of the primary template that the \grammar{template-name} designates.
	\item \field{arguments} denotes the angle bracket-enclosed comma-separated sequence of template arguments.
	\item \field{locus} denotes the source location of this syntax tree.
	\item \field{template}, when not null, indicates the source location of the \code{template} keyword if used in forming the \grammar{template-id} (typically as part of a \grammar{qualified-id}).
\end{itemize}

\partition{syntax.template-id}


\subsection{\valueTag{SyntaxSort::MemInitializer}}
\label{sec:ifc:SyntaxSort:MemInitializer}

\partition{syntax.mem-initializer}


\subsection{\valueTag{SyntaxSort::CtorInitializer}}
\label{sec:ifc:SyntaxSort:CtorInitializer}

\partition{syntax.ctor-initializer}
		
		
\subsection{\valueTag{SyntaxSort::LambdaIntroducer}}
\label{sec:ifc:SyntaxSort:LambdaIntroducer}

\partition{syntax.lambda-introducer}

\subsection{\valueTag{SyntaxSort::LambdaDeclarator}}
\label{sec:ifc:SyntaxSort:LambdaDeclarator}

\partition{syntax.lambda-declarator}

\subsection{\valueTag{SyntaxSort::CaptureDefault}}
\label{sec:ifc:SyntaxSort:CaptureDefault}

\partition{syntax.capture-default}

\subsection{\valueTag{SyntaxSort::SimpleCapture}}
\label{sec:ifc:SyntaxSort:SimpleCapture}

\partition{syntax.simple-capture}

\subsection{\valueTag{SyntaxSort::InitCapture}}
\label{sec:ifc:SyntaxSort:InitCapture}

\partition{syntax.init-capture}

\subsection{\valueTag{SyntaxSort::ThisCapture}}
\label{sec:ifc:SyntaxSort:ThisCapture}

\partition{syntax.this-capture}

\subsection{\valueTag{SyntaxSort::AttributedStatement}}
\label{sec:ifc:SyntaxSort:AttributedStatement}

\partition{syntax.attributed-statement}

\subsection{\valueTag{SyntaxSort::AttributedDeclaration}}
\label{sec:ifc:SyntaxSort:AttributedDeclaration}

\partition{syntax.attributed-declaration}

\subsection{\valueTag{SyntaxSort::AttributeSpecifierSeq}}
\label{sec:ifc:SyntaxSort:AttributeSpecifierSeq}

\partition{syntax.attribute-specifier-seq}


\subsection{\valueTag{SyntaxSort::AttributeSpecifier}}
\label{sec:ifc:SyntaxSort:AttributeSpecifier}

\partition{syntax.attribute-specifier}

\subsection{\valueTag{SyntaxSort::AttributeUsingPrefix}}
\label{sec:ifc:SyntaxSort:AttributeUsingPrefix}


\partition{syntax.attribute-using-prefix}

\subsection{\valueTag{SyntaxSort::Attribute}}
\label{sec:ifc:SyntaxSort:Attribute}

\partition{syntax.attribute}


\subsection{\valueTag{SyntaxSort::AttributeArgumentClause}}
\label{sec:ifc:SyntaxSort:AttributeArgumentClause}

\partition{syntax.attribute-argument-clause}

\subsection{\valueTag{SyntaxSort::Alignas}}
\label{sec:ifc:SyntaxSort:Alignas}

\partition{syntax.alignas}


\subsection{\valueTag{SyntaxSort::UsingDeclaration}}
\label{sec:ifc:SyntaxSort:UsingDeclaration}

\partition{syntax.using-declaration}


\subsection{\valueTag{SyntaxSort::UsingDeclarator}}
\label{sec:ifc:SyntaxSort:UsingDeclarator}

\partition{syntax.using-declarator}


\subsection{\valueTag{SyntaxSort::UsingDirective}}
\label{sec:ifc:SyntaxSort:UsingDirective}

\partition{syntax.using-directive}


\subsection{\valueTag{SyntaxSort::ArrayIndex}}
\label{sec:ifc:SyntaxSort:ArrayIndex}

\partition{syntax.array-index}

\subsection{\valueTag{SyntaxSort::SEHTry}}
\label{sec:ifc:SyntaxSort:SEHTry}

\partition{syntax.seh-try}

\subsection{\valueTag{SyntaxSort::SEHExcept}}
\label{sec:ifc:SyntaxSort:SEHExcept}

\partition{syntax.seh-except}

\subsection{\valueTag{SyntaxSort::SEHFinally}}
\label{sec:ifc:SyntaxSort:SEHFinally}

\partition{syntax.seh-finally}

\subsection{\valueTag{SyntaxSort::SEHLeave}}
\label{sec:ifc:SyntaxSort:SEHLeave}

\partition{syntax.seh-leave}

\subsection{\valueTag{SyntaxSort::TypeTraitIntrinsic}}
\label{sec:ifc:SyntaxSort:TypeTraitIntrinsic}

\partition{syntax.type-trait-intrinsic}


\subsection{\valueTag{SyntaxSort::Tuple}}
\label{sec:ifc:SyntaxSort:Tuple}

A \type{SyntaxIndex} value with tag \valueTag{SyntaxSort::Tuple} represents a reference
to sequence or more abstract indices to syntax trees. 
The \field{index} field is index into the tuple syntax tree partition.
Each entry in that partition is a structure with the following layout
%
\begin{figure}[H]
	\centering
	\structure{
		\DeclareMember{start}{Index} \\
		\DeclareMember{cardinality}{Cardinality} \\
	}
	\caption{Structure of a tuple syntax tree}
	\label{fig:ifc-tuple-syntax-tree-structure}
\end{figure}
%
with the following meanings for the fields:
\begin{itemize}
	\item field{start} is an index into the syntax tree heap partition. It points to the first syntax tree abstract reference in the tuple.
	\item \field{cardinality} denotes the number of syntax tree abstract references in the tuple.
\end{itemize}

\partition{syntax.tuple}
	
	
\subsection{\valueTag{SyntaxSort::AsmStatement}}
\label{sec:ifc:SyntaxSort:AsmStatement}

\partition{syntax.asm-statement}

\subsection{\valueTag{SyntaxSort::NamespaceAliasDefinition}}
\label{sec:ifc:SyntaxSort:NamespaceAliasDefinition}

\partition{syntax.namespace-alias-definition}

\subsection{\valueTag{SyntaxSort::Super}}
\label{sec:ifc:SyntaxSort:Super}

\partition{syntax.super}

\subsection{\valueTag{SyntaxSort::UnaryFoldExpression}}
\label{sec:ifc:SyntaxSort:UnaryFoldExpression}

\partition{syntax.unary-fold-expression}


\subsection{\valueTag{SyntaxSort::BinaryFoldExpression}}
\label{sec:ifc:SyntaxSort:BinaryFoldExpression}

\partition{syntax.binary-fold-expression}


\subsection{\valueTag{SyntaxSort::EmptyStatement}}
\label{sec:ifc:SyntaxSort:EmptyStatement}

\partition{syntax.empty-statement}
		
\subsection{\valueTag{SyntaxSort::StructuredBindingDeclaration}}
\label{sec:ifc:SyntaxSort:StructuredBindingDeclaration}

\partition{syntax.structured-binding-declaration}
		
\subsection{\valueTag{SyntaxSort::StructuredBindingIdentifier}}
\label{sec:ifc:SyntaxSort:StructuredBindingIdentifier}

\partition{syntax.structured-binding-identifier}

\subsection{\valueTag{SyntaxSort::UsingEnumDeclaration}}
\label{sec:ifc:SyntaxSort:UsingEnumDeclaration}

\partition{syntax.using-enum-declaration}





